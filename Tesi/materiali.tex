\chapter{Strumenti matematici}
\label{chap:mat}

\begin{minipage}{12cm}\textit{In questo capitolo verranno brevemente introdotti gli strumenti che verranno utilizzati in questo lavoro di tesi. Si introdurranno per prima gli strumenti di geometria algebrica quali: ideali, basi di Groebner, teoria dell'eliminazione. Successivamente si procederà ad illustrare la decomposizione SVD.}
\end{minipage}

\vspace*{1cm}

\section{Geometria algebrica}
\label{sec:Geom}

Per comprendere al meglio il lavoro svolto, conviene introdurre alcuni strumenti algebrici fondamentali. Si introdurranno le definizioni algebriche principali e successivamente si introdurrà il concetto di base di Groebner. 

\begin{defn}Si definisce \textbf{monomio} nelle variabili \textit{$x_1, ..., x_n$} un prodotto del tipo $x_1^{\alpha_1} \cdot ... \cdot x_n^{\alpha_n}$ dove gli $\alpha_i$ sono interi non negativi. Si utilizza inoltre la seguente notazione in forma vettoriale \textbf{$x^\alpha$}, dove $x = [x_1 \dots x_n]^T$ e $\alpha = [\alpha_1 \dots \alpha_n]^T$ 	
\end{defn}

\begin{defn}Si definisce \textbf{polinomio} \textit{p} con i coefficienti del campo $\mathbb{K}$ una combinazione $\mathbb{K}$-lineare finita di monomi del tipo
	\begin{center}
		$p = \sum_{\alpha}^{} a_\alpha x^{\alpha}$, $a_\alpha \in \mathbb{K}$
	\end{center}
	si usa indicare con \textbf{termine} l'elemento $\alpha$-esimo della sommatoria.	
\end{defn}

\begin{defn}
	Si usa indicare con $\mathbb{K}[x]$ l'insieme di tutti i polinomi che è possibile generare con monomi nelle variabili $x = [x_1 \dots x_n]^T$ e i coefficienti in $\mathbb{K}$.
\end{defn}

\begin{defn}
	Dato un set di polinomi $p_1(x), ...m p_s(x) \in \mathbb{K}$ si definisce \textbf{varietà affine} il seguente set
	\begin{center}
		$V(p_1(x), ..., p_s(x)) := \{x \in \mathbb{K}^n  \; | \; p_i(x) = 0, i = 1, ..., s \}$
	\end{center} 
\end{defn}

Date due varietà affini $V_1(p_1, ..., p_s)$ e $V_2(q_1, ..., q_t)$ è possibile definire le operazioni di unione e intersezione:
\begin{align}
	\nonumber
	& V_1 \cup V_2 = V(p_1q_1, ...,p_1q_t, p_2q_1, ..., p_2q_t, ..., p_sq_1, ..., p_sq_t), \\
	\nonumber
	& V_1 \cap V_2 = V(p_1, ..., p_s, q_1, ..., q_t).
\end{align}
Si può ora introdurre lo strumento fondamentale utilizzato e alcune sue proprietà.
	
\begin{defn}
	Si definisce \textbf{ideale} \textit{I} un subset di $\mathbb{K}[x]$ che gode delle seguenti proprieta:
	\begin{enumerate}
		\item $0 \in I $,
		\item $f, g \in I \Rightarrow f + g \in I$,
		\item $f \in I, h \in \mathbb{K}[x] \Rightarrow fh \in I$
	\end{enumerate}
\end{defn}
Un modo naturale di definire un ideale è generarlo a partire da un numero finito di polinomi $p_1, ..., p_s$:
\begin{center}
	$\left\langle p_1, ..., p_s \right\rangle := \left\lbrace p \in \mathbb{K}[x] \; | \; p = \sum_{i = 1}^{s} h_ip_i, h_i \in \mathbb{K}[x] \right\rbrace $
\end{center}
Siano $p_1, ..., p_s \in \mathbb{K}[x]$, se si considera un ideale $I = \left\langle p_1, ..., p_s \right\rangle$ esso può essere interpretato nella seguente maniera. Si considerino le seguenti equazioni,
\begin{align}
\nonumber
& p_1 = 0 \\
\nonumber
& \vdots\\
\nonumber
& p_s = 0
\end{align}
se considero $h_1, ..., h_s \in \mathbb{K}[x]$ tali equazioni \textbf{avranno come conseguenza} che
\begin{center}
	$h_1p_1 + ... + h_sp_s = 0$.
\end{center}
Si evince che si può considerare un ideale come l'insieme di tutte le conseguenze generate dal sistema di equazioni omogeneo formato dai polinomi generatori dell'ideale.
\begin{prop}
	Considerati i polinomi $p_1, ..., p_s$ e $q_1, ..., q_r \in \mathbb{K}[x]$ si ha che
	\begin{center}
		$\left\langle p_1, ..., p_s \right\rangle = \left\langle q_1, ..., q_r \right\rangle \Leftrightarrow V(p_1, ..., p_s) = V(q_1, ..., q_r)$
	\end{center}
\end{prop}
Dalla suddetta proprietà si evince che una varietà affine non viene alterata da un cambio di base, cioè essa è definita dall'ideale stesso e non dai polinomi usati come generatori.
\begin{defn}
	Data una varietà affine $V$ si definisce come l'\textbf{ideale di \textit{V}} il set di tutti i polinomi in $\mathbb{K}[x]$ che si annullano su di essa. In formule:
	\begin{center}
		$\mathbf{I}(V) := \left\lbrace p \in \mathbb{K}[x] \; | \; p(x) = 0, \quad \forall x \in V \right\rbrace $
	\end{center}
\end{defn}
In generale vale la seguente proprietà:
\begin{prop}
	Siano $p_1, ..., p_s \in \mathbb{K}[x]$ allora vale che $\left\langle p_1, ..., p_s \right\rangle \subseteq \mathbf{I}(V(p_1, ..., p_s))$. In generale il converso non vale.
\end{prop}
Si può definire infine il concetto di varietà affine di un ideale.
\begin{defn}
	Sia $I$ un ideale in $\mathbb{K}[x]$ il set
	\begin{center}
		$V(I) := \left\lbrace x \in \mathbb{K}^n \; | |; p(x) = 0, \forall p \in I \right\rbrace$
	\end{center}
	si dice \textbf{varietà affine dell'ideale} \textit{I}.
\end{defn}
Vale la seguente proprietà che sarà particolarmente utile alla definizione e determinazione delle preannunciate basi di Groebner.
\begin{prop}
	Sia \textit{I} un ideale in $\mathbb{K}[x]$ e siano $p_1, ..., p_s$ polinomi, allora vale che $V(I) = V(p_1, ..., p_s)$ per ogni base $\left\lbrace p_1, ..., p_s \right\rbrace$ di $I$
\end{prop}

Si consideri un polinomio nella variabile scalare s, del tipo $p = s^n + a_{n-1}s^{n-1} + \cdots + a_1s + a_0$, se considero un ulteriore polinomio $g \in \mathbb{K}[s]$, di grado $k \geq n$ è possibile dimostrare che esso è esprimibile con una relazione del tipo $g = qp + r$, dove $q$ è detto polinomio quoziente e $r$ resto. Inoltre vale che $deg(r) \le deg(g)$. In letteratura esistono diversi algoritmi volti a trovare i polinomio $q$ e $r$, ad esempio l'algoritmo di divisone tra polinomi che estende il concetto di divisione tra scalari.

Un fatto importante da notare è che se si considera un ideale $I = \left\langle g \right\rangle$, dove $g \in \mathbb{K}[x]$, è possibile allora verificare in modo semplice se un polinomio $p \in \mathbb{K}[x]$ appartiene all'ideale $I$ verificando che il resto della divisione di $p$ da $g$ è nullo.

Tale concetto, esteso, porta alla definizione delle suddette basi di Groebner e quindi alla determinazione di una base di ordine ridotto, in senso che verrà presto definito, che definisce un determinato ideale $I$. Al fine di poter estendere tale concetto è necessario poter definire un ordinamento totale tra i polinomi. Nel caso ad una sola variabile, un ordinamento naturale è quello di considerare il grado dei polinomi (l'esponente maggiore con cui compare la variabile del polinomio). Si può estendere tale approccio ma occorre stabilire un ordinamento lessicografico tra i monomi di cui un polinomio si compone. Tale ordinamento può essere definito in diversi modi, nel nostro caso useremo il seguente.

\begin{defn}
	Si considerino due monomi $x^{\alpha}$ e $x^{\beta}$, dove $x = [x_1 \; ... \; x_n]^T$ e $\alpha, \beta \in \mathbb{Z}^n$, si dice che $x^{\alpha}$ e $x^{\beta}$ sono \textbf{in ordine lessicografico}, in simboli $x^{\alpha} \ge_{Lex} x^{\beta}$, con $x_1 \ge x_2 \ge \cdots \ge x_n$ se vale che
	il primo elemento non nullo del vettore $\alpha-\beta$ è maggiore di zero.	
\end{defn}
Si fa notare che il precedente ordinamento è fortemente influenzato dall'ordinamento che si da alle variabili $x_i$, infatti al variare di tale ordinamento (e corrispettiva variazione dell'ordinamento sui vettori $\alpha$ e $\beta$) si ottengono ordinamenti diversi.

\begin{defn}
	Dato un polinomio $p \in \mathbb{K}[x]$ se si definisce un ordine lessicografico e si riscrive tale polinomio in modo ordinato, $ p = a_1x^{\alpha_1} + a_2x^{\alpha_2} + \cdots $,  si definiscono il \textbf{leading monomial} $LM(p) := x^{\alpha_1}$, il \textbf{leading coefficient} $LC(p) := a_1$ e il \textbf{leading terms} $LT(p) := a_1x^{\alpha_1}$
\end{defn}
Si può estendere quindi il concetto di divisione tra polinomi, al fine di ottenere un algoritmo multi-variabile in grado di dividere un polinomio per un set di polinomi. In altre parole dato un set di polinomi $p_1, ..., p_s$ e un polinomio $p$ appartenenti a $\mathbb{K}[x]$ è possibile trovare $q_1, ... , q_s, r \in \mathbb{K}[x]$ tali che $p = q_1p_1 + q_1p_2 + \cdots + q_sp_s + r$ e vale che o $r = 0$ oppure $r = c_1x^{\beta_1} + c_2x^{\beta_2} + \cdots $ e nessun $c_ix^{\beta_i}$ è divisibile per nessun $LT(p_j)$ con $i = 1,2,..., j = 1, 2,..., s$.

\begin{obs}
	Si può quindi facilmente capire che dato un ideale $I = \left\langle p_1, ..., p_s \right\rangle, p_i \in \mathbb{K}[x]$ e un ulteriore polinomio $p \in \mathbb{K}[x]$, quest'ultimo diviso per il set suddetto, si otterrà un resto nullo se e solo $ p \in I$. In altre parole si è trovato uno strumento per verificare se un polinomio è conseguenza di un ideale.
\end{obs}

Si riporta in seguito un algoritmo, chiamato \textbf{Multivariate division algoritmh}, atto a tale scopo. \\

\begin{algorithm}[H]
	\SetAlgoLined
	\KwData{una tupla ordinata $p_1, ..., p_s$ di polinomi $\in \mathbb{K}[x]$ e un polinomio $p \in \mathbb{K}[x]$}
	\KwResult{$q_1, ..., q_s, r \in \mathbb{K}[x]$}
	set $q_1 := 0, ..., q_s := 0$\;
	set r := 0\;
	set f := p\;
	\While{f $\neq$ 0}{
		i = 1; divisionOccurred = false\;
		\While{$i \leq s$ AND divisionOccurred == False}{
			\eIf{LT($p_i$) divide LT(f)}{
				$q_i = q_i + \frac{LT(f)}{LT(p_i)}$\;
				$f = f - \frac{LT(f)}{LT(p_i)}p_i$\;
				divisionOccurred = True\;
			}{
				$i = i + 1$\;
			}
		}
		\If{divisionOccurred == False}{
			$r = r + LT(f) $\;
			$f = f - LT(f) $\;
		}
	}
	\caption{Multivariate division algoritmh}
\end{algorithm}

\begin{prob}
	\label{prob:mvda}
	Il risultato del precedente algoritmo è fortemente dipendente dall'ordinamento dato ai polinomi $p_i$. Si vogliono trovare, se esistono, strumenti tali da rendere il calcolo del resto di tale divisione indipendente dall'ordinamento dato al set dei polinomi $p_i$.
\end{prob}

Il problema \ref{prob:mvda} costituisce la motivazione per introdurre le preannunciate basi di Groebner (ridotte), le quali, come si vedrà, permettono di risolvere tale problema.

\begin{defn}
	Si consideri l'ideale $I \subset \mathbb{K}[x]$ non vuoto, si fissi un ordinamento per i monomi e si definisca l'insieme
	\begin{center}
		$LT(I) := \left\lbrace cx^{\alpha} monomio \; | \; \exists f \in I \; \; \; t.c. \; \;  LT(f) = cx^{\alpha} \right\rbrace $
	\end{center} 
	e l'ideale dei leading terms generato da tale insieme $\left\langle LT(I) \right\rangle $.
	Un subset finito $G := \left\langle g_1, ..., g_m\right\rangle \subset I$ si dice una \textbf{base di Groebner} per l'ideale $I$ se vale
	\begin{center}
		$\left\langle LT(g_1), ..., LT(g_m) \right\rangle = \left\langle LT(I) \right\rangle$
	\end{center}
\end{defn}

\begin{prop}
	Sia $\left\langle \O \right\rangle := \{0\}$ il subset vuoto e sia esso per convenzione una base di Groebner per l'ideale vuoto, allora per ogni ideale esiste (non necessariamente unica) una base di Groebner che è una base per tale ideale.
\end{prop}

\begin{defn}
	Un polinomio $r \in \mathbb{K}[x]$ si dice \textbf{ridotto} rispetto ad un set $\left\lbrace g_1, ..., g_m \right\rbrace$, $g_i \in \mathbb{K}[x]$, se esso è esprimibile come una combinazione $\mathbb{K}$-lineare di termini nessuno dei quali è divisibile per nessuno dei $LT(g_1), ..., LT(g_m)$.
\end{defn}

\begin{defn}
	Una base di Groebner $\left\lbrace g_1, ..., g_m \right\rbrace$ si dice \textbf{ridotta} se vale che:
	\begin{enumerate}
		\item $LC(g_i) = 1, i = 1, ..., m$,
		\item $g_i$ è ridotto rispetto al set $\left\lbrace g_1, ...g_{i-1},g_{i+1}, ..., g_m \right\rbrace, i = 1, ..., m$
	\end{enumerate}
\end{defn}

\begin{prop}
	Per ogni ideale $I \subset \mathbb{K}[x]$ esiste un unica base di Groebner ridotta.
\end{prop}

\begin{prop}
	Sia $p \in \mathbb{K}[x]$ e $G$ una base di Groebner per l'ideale $I \subset \mathbb{K}[x]$. Allora il resto della divisione di $p$ per il set $G$ è unico indipendentemente dall'ordine dei polinomi $g_i$.
\end{prop}
Risulta abbastanza chiaro che l'uso delle basi di Groebner per definire un ideale permette di risolvere il problema \ref{prob:mvda}; infatti un polinomio $p \in I$ se e solo se vale che il resto della divisione tra $p$ e la base di Groebner di $I$ è nullo.
Si pone a questo punto il problema della determinazione di una base di Groebner dato un ideale I. A tal fine si può utilizzare l'algoritmo di Buchberger.

\begin{algorithm}[H]
	\SetAlgoLined
	\KwData{un set $p_1, ..., p_s$ di polinomi $\in \mathbb{K}[x]$}
	\KwResult{una base di Groebner $G$ per $\left\langle p_1, ..., p_s \right\rangle $}
	$G = \left\lbrace p_1, ..., p_s \right\rbrace $\;
	\Repeat{$G == $\^{G}}{
		\^{G} = G\;
		\ForEach{$p, q \in$ \^{G}, $p \neq q$}{
			$x^{\alpha} = LM(p)$; $x^{\beta} = LM(q)$\;
			$\gamma = (\gamma_1, ..., \gamma_n)$; $\gamma_i = max(\alpha_i, \beta_i)$\;
			$s = \frac{x^{\gamma}}{LT(p)}p - \frac{x^{\gamma}}{LT(q)}q$\;
			r := resto della divisione tra $s$ e il set \^{G}\;
			\If{$r \neq 0$}{
				$G = G \cup {r}$\;
			}
		}
	}
	
	\caption{Buchberger's algoritmh}
\end{algorithm}

\medskip
Si noti che il suddetto algoritmo non fornisce in genere una base di Groebner ridotta, infatti mantiene in G i polinomi $p_i$. 
Si fa notare che lo scopo del polinomio $s$ è quello di permettere cancellazioni tra i leading terms dei polinomi $p$ e $q$, ci si riferisce ad esso come il polinomio-S, in simboli $S(p,q)$. Infine la condizione di terminazione è corretta perché è possibile dimostrare che un set di polinomi $G = \{g_1, ..., g_m\}$ è una base di Groebner per un certo ideale $I$ se e solo se per ogni $i,j = 1,..., m , \; i\neq j$ il resto della divisione di $S(g_i, g_j)$ per il set $G$ è nullo (Buchberger's criterion). Si fa presto a capire che siccome per ogni coppia (p,q) si deve effettuare una divisione con tutti i polinomi in \^{G} ed inoltre per ogni iterazione si aggiungerà un polinomio a tale set, l'algoritmo risulta pesante dal punto di vista computazionale. Infatti, si può dimostrare che l'algoritmo è EXPSPACE-complesso, cioè con tempo di calcolo e occupazione di memoria esponenziali. In genere però esso risulta applicabile sui computer moderni.
Nel seguito della trattazione si utilizzerà il software Macaulay2 per il calcolo delle basi di Groebner di un dato ideale, il quale implementa proprio il suddetto algoritmo.


% --------- elimination theory
\subsubsection{Teoria dell'eliminazione}
Durante gli studi algebrici sicuramente ci si può essere imbattuti in un noto algoritmo detto: \textbf{Eliminazione di Gauss}. Quest'ultimo permette, dato un sistema lineare di equazioni - rappresentabile attraverso una matrice, attraverso operazioni elementari su sopraddetta matrice (operazioni che non ne alterano il rango), di ottenere una rappresentazione, in ipotesi di rango riga pieno, detta (quasi-)triangolare superiore. In tal modo si ottiene un sistema di equazioni in cui le ultime equazioni non dipendono da un certo numero delle prime variabili. Selezionando un certo numero delle ultime equazioni, è possibile ottenere un sotto-sistema in cui le prime variabili non compaiono (sono state eliminate). 
Tale approccio viene utilizzato per risolvere in maniera più semplice un sistema di equazioni, in ipotesi che possa essere fatto.

Tale concetto può essere esteso al caso polinomiale, non lineare, multi-variabile.

\begin{defn}
	Un set di polinomi $p_1, ..., p_m$, $p_i \in \mathbb{K}[x]$, si dice linearmente dipendente se esistono $c_1, ... c_m$ costanti non tutte nulle tale che $c_1p_1 + c_2p_2 + \, ... \, + c_mp_m = 0$. Altrimenti si dice \textbf{linearmente indipendente}.
	
	Un set di polinomi $p_1, ..., p_m$, $p_i \in \mathbb{K}[x]$, si dice algebricamente dipendente se esiste $q \in \mathbb{K}[y_1, \, ... \, , y_m]$ polinomio non nullo, tale che  $q(p_1, \, ..\, p_m) = 0 \in \mathbb{K}[x]$ (è il polinomio nullo). Altrimenti si dice \textbf{algebricamente indipendente}.
\end{defn}

Dato un ordinamento lessicografico $x_1 \ge_{Lex} x_2 \ge_{Lex} ... \ge_{Lex} x_n$ può essere possibile eliminare le prime $n-1$ variabili. Si dice infatti che tale ordine \textbf{elimina} $x_1, \, ... \, , x_{n-1}$.

\begin{defn}
	Dato un ordinamento lessicografico $x_1 \ge_{Lex} x_2 \ge_{Lex} ... \ge_{Lex} x_n$ e un ideale $I \subset \mathbb{K}[x]$ allora l'ideale
	\begin{center}
		$I_l = I \cap \mathbb{K}[x_{l+1}, \, ... \, x_{n}]$
	\end{center}
	si dice l'ideale di eliminazione di ordine $l$.
\end{defn} 
L'ideale di eliminazione di ordine $l$ può essere visto come l'insieme di tutte le conseguenze di un ideale nelle quali le prime $l$ variabili non compaiono.

Si può introdurre infine il teorema generale di eliminazione.
\begin{teor}
	\label{teo:elimination}
	Sia dato un ideale $I \subset \mathbb{K}[x_a, x_b]$ con $x_a \in \mathbb{R}^n_a, x_b \in \mathbb{R}^n_b$, $ n_a, n_b > 0$, $n = n_a + n_b$, e un ordinamento lessicografico $x_a \ge_{Lex} x_b$. Inoltre sia $G$ una base di Groebner ridotta per l'ideale $I$, calcolata in accordo a tale ordinamento. Allora vale che una base di Groebner per l'ideale di eliminazione di ordine $n_a$ è 
	\begin{center}
		$G_{n_a} = G \cap \mathbb{K}[x_b]$.
	\end{center}
\end{teor}

Mediante l'utilizzo delle basi di Groebner e il precedente teorema è possibile quindi estendere il concetto di eliminazione, per applicarlo a sistemi di equazioni omogenei multi-variabile polinomiali non lineari. In modo di ottenere delle conseguenze, sistemi omogenei di dimensione ridotta dello stesso tipo, che non contengano alcune variabili. Tale strumento può essere utilizzato per ottenere soluzioni analitiche ai problemi studiati. 

\section{Decomposizione ai valori singolari}
\label{sec:SVD}

In algebra lineare, la decomposizione ai valori singolari, detta anche SVD (dall'acronimo inglese Singular Value Decomposition), è una particolare fattorizzazione di una matrice basata sull'uso di autovalori e autovettori. 

\begin{defn}
	Data una matrice $M \in \mathbb{C}^{m \times n}$, si definisce decomposizione ai valori singolari o SVD la seguente decomposizione:
	\begin{center}
		$M = U \Sigma V^*$
	\end{center}
	dove $U$ è una matrice unitaria di dimensioni  $m\times m$, $\Sigma$ è una matrice diagonale rettangolare di dimensioni $m\times n$ e $V^*$ è la trasposta coniugata di una matrice unitaria $V$ di dimensioni $n\times n$.
	Si definiscono inoltre:
	\begin{enumerate}
		\item gli elementi di $\Sigma$ valori singolari della matrice $M$,
		\item le colonne di $U$ vettori singolari sinistri della matrice $M$,
		\item le righe di $V^*$ covettori singolari destri  della matrice $M$.		
	\end{enumerate}
\end{defn}

\begin{prop}
	Data una decomposizione SVD valgono le seguenti proprietà:
	\begin{enumerate}
		\item se $ m \le n$, allora $\Sigma = [\Sigma_0 \, \, 0]$, dove $\Sigma_0 = diag \{\sigma_1, ..., \sigma_m \}, \sigma_i \in \mathbb{R}, \sigma_i \ge 0$,
		\item se $ m > n$, allora $\Sigma = [\Sigma_0^T \, \, 0^T]^T$, dove $\Sigma_0 = diag \{\sigma_1, ..., \sigma_n \}, \sigma_i \in \mathbb{R}, \sigma_i \ge 0$,	 
		\item i valori singolari $\sigma_i$ non nulli di $M$ sono le radici quadrate degli autovalori della matrice $MM^*$ e $M^*M$,
		\item i vettori singolari di sinistra di $M$ sono gli autovettori di $MM^{*}$,
		\item i covettori singolari di destra di $M$ sono gli autovettori di $M^{*}M$ trasposti.
	\end{enumerate}
\end{prop}

Per il calcolo della decomposizione SVD sebbene è possibile procedere usando la definizione data, questo è sconsigliato perché inefficiente. Negli ultimi anni sono stati sviluppati diversi algoritmi atti allo scopo. Nello specifico si cita l'algoritmo Golub/Kahan utilizzato nelle recenti librerie matematiche come LAPACK, BLAS e CUBLAS.

In particolare si fa notare che questi algoritmi sono particolarmente efficienti quando la matrice da decomporre è quadrata.

La fattorizzazione SVD ha numerose applicazioni pratiche, infatti:
\begin{enumerate}
	\item il rango di una matrice è pari al numero di valori singolari maggiori di zero nella sua decomposizione SVD.
	\item può essere usata per il calcolo della matrice pseudo-inversa destra (o sinistra). Infatti sia $M \in \mathbb{C}^{m \times n}, rank(M) = m$ allora la sua decomposizione sarà $M = U [ \Sigma_0 \,\, 0 ] V^*, \Sigma_0 > 0, UU^* = I, VV^* = I$, allora una pseudo inversa destra per $M$ può essere calcolata come:
	\begin{center}
		$M^{R} = V 
		\begin{bmatrix}
			\Sigma_0^{-1} \\
			Z
		\end{bmatrix} U^*$
	\end{center} 
	dove $Z$ è una qualsiasi matrice che può essere anche nulla e $\Sigma_0^{-1}$ è sempre diagonale e cui elementi saranno del tipo $1/\sigma_i$.
	\item come vedremo nel paragrafo \ref{sec:kabsch} è particolarmente utile per risolvere problemi ai minimi quadratici.
\end{enumerate}


\section{Rotazioni}