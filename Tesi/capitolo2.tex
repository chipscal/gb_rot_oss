\chapter{Programmi utili}

\begin{minipage}{12cm}\textit{In questo capitolo verr\`a esposto un breve elenco con i link ai programmi pi\`u utili per scrivere nel formato LaTeX.}
\end{minipage}

\vspace*{1cm}

\section{Linux}

Le maggiori distribuzioni di Linux comprendono gi\`a al loro interno una distribuzione di LaTeX. In caso contrario si faccia una ricerca all'interno della documentazione relativa per scoprire come installarla (es.: in Gentoo basta un ``emerge tetex'').\\

Riguardo gli ambienti di sviluppo di consiglia Kile (kile.sourceforge.net), uno dei migliori software in circolazione per la scrittura LaTeX (quello con cui \`e anche stato redatto questo documento). In alternativa (se si \`e pi\`u esperti) \`e possibile anche usare Vim, che contiene gi\`a al suo interno l'evidenziazione della sintassi LaTeX.


\section{Windows}

In ambiente Windows \`e posibile utilizzare LaTeX attraverso la distribuzione MikTeX (www.miktex.org), che garantisce la piena compatibilit\`a con il sistema TeX di Linux. Tale ambiente \`e inoltre dotato di vari strumenti di configurazione grafica che lo rendono di semplice utilizzo.\\

Per scrivere file TeX si consiglia il software TeXnicCenter (www.toolscenter.org), dotato di evidenziazione della sintassi e dell'inserimento facilitato di parecchi comandi LaTeX. Inoltre si suggerisce anche TexAide (www.dessci.com/en/products/texaide), un programma che permette di comporre visualmente formule matematiche (nello stile dell'Equation Editor di Microsoft) e quindi tradurle in formato LaTeX.