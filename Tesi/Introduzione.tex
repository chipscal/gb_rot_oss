\chapter*{Ringraziamenti}
\addcontentsline{toc}{chapter}{Ringraziamenti}
Un ringraziamento va innanzitutto al mio relatore, Daniele Carnevale che mi ha guidato durante questo percorso formativo, spronandomi a dare il massimo.
Vorrei ringraziare poi Corrado Possieri per il prezioso aiuto durante lo sviluppo ed i pomeriggi spesi insieme. 
\newline

Vorrei ringraziare i miei compagni di corso per le giornate passate insieme, i progetti un pò folli che tutto sommato ci hanno permesso di imparare e crescere insieme.
\newline

Un altro grazie va di certo ai miei genitori ed ai miei fratelli che mi hanno sostenuto e incoraggiato. 
\newline

Ed infine un ringraziamento speciale va alla mia ragazza, che in questi anni è
stata la mia forza, ha dovuto sopportare i miei "sproloqui" riguardanti il progetto di turno ed è stata presente nei momenti difficili. 


\chapter*{Introduzione}
\addcontentsline{toc}{chapter}{Introduzione}

Questo lavoro di tesi nasce, inizialmente, da una richiesta di collaborazione da parte di Thales Alenia. In tale collaborazione si richiedeva lo studio, ed eventualmente la sintesi di una nuova soluzione ad un problema di stima. Quest'ultimo è stato così posto: \textbf{si considerino diverse misure della velocità angolare di un corpo rigido nello spazio, misure effettuate da due sensori differenti posti in posizioni e orientamenti diversi, si richiede, se possibile, di individuare la matrice di rotazione relativa tra il primo ed il secondo. Si deve essere in grado cioè di individuare l'orientamento del secondo sensore rispetto al primo.} Tale problema risulta particolarmente rilevante in ambito satellitare. Spesso infatti, in seguito alla messa in orbita di un satellite, quest'ultimo può subire deformazioni permanenti o transitorie. I sensori a bordo perciò possono risultare ruotati rispetto all'orientamento nominale e fornire una misura di velocità angolare scorretta. Se si individua però la trasformazione avvenuta, si è in grado di correggere la misura.

La prima metà di questo lavoro di tesi si concentrerà sulla soluzione del suddetto problema. Si preannuncia che la domanda posta ha risposta affermativa e si forniranno due soluzioni. In realtà tali soluzioni abbracceranno un problema più ampio, che comprende il primo, nel quale si richiede la stima di una trasformazione completa: matrice di rotazione e vettore di traslazione. Quest'ultimo, nel constesto su detto, potrebbe identificare un bias tra i due sensori. 
La prima soluzione si basa sull'utilizzo di tecniche di \textbf{Geometria algebrica} ed in particolare delle \textbf{basi di Groebner}. La seconda è un adattamento di un algoritmo spesso utilizzato in bio-informatica per il confronto di proteine, detto \textbf{Algoritmo di Kabsch} e garantirà una maggiore robustezza ai rumori in misura.

La seconda metà di questo lavoro invece nasce per curiosità personale ed abbraccia il settore della \textbf{visione artificiale}. Comprese le potenzialità degli strumenti individuati infatti, si è scelto di applicarli per fornire una soluzione ad un problema molto rilevante in contesti di \textbf{robotica mobile}. Si è pertanto progettato, implementato e testato un sistema di stima del posizionamento e orientamento assoluti di un osservatore mobile rispetto all'ambiente circostante, eventualemente ignoto, denominato \textbf{Visual Observer}. Tale strumento fa uso solo di fotocamere (almeno due) e può essere utilizzato in ambienti dove altri tipi di sensori falliscono o in concomitanza a soluzioni più comuni al fine di migliorare le prestazioni di stima. Ad esempio potrebbe essere utilizzato in ambienti al chiuso per calcolare la posizione di un drone.

Nel capitolo \ref{chap:mat} si introdurranno brevemente gli strumenti matematici utilizzati che si suppone possano essere non noti al lettore perché non trattati nel corso di studi.
Nel capitolo \ref{chap:visualObs} si motiverà e si esplicherà il funzionamento del sistema di posizionamento. Si renderà necessario, come vedremo, sia risolvere il problema di stima vero e proprio sia la generazione di un input valido per gli algoritmi individuati. Si preannuncia che sarà necessario: riuscire a misurare le coordinate dei punti del mondo utilizzando tecniche di \textbf{steroscopia}, scegliere un set di punti del mondo e "ricercare" gli stessi in un istante successivo al netto dello spostamento, mediante tecniche di \textbf{Feature Detection}, per procedere alla generazione di un set di coppie di "misure" da utilizzare come input agli algoritmi di stima individuati. Nel capitolo \ref{chap:stima} si risolverà pertanto il primo problema, invece nel \ref{chap:visione} si provvederà a fornire gli strumenti fini alla generazione del suddetto set.

Infine nel capitolo \ref{chap:implTest} si illustreranno le problematiche riscontrate, gli strumenti utilizzati e le soluzioni individuate durante l'implementazione del software. L'intera implementazione sarà improntata nel rispetto di un requisito di esecuzione realtime, si di poter rendere il sistema utilizzabile in contesti di controllo in feedback di robot mobili. Infine si testeranno le prestazioni e si validerà il sistema mediante l'uso di un motore grafico e l'interazione quindi con un ambiente virtuale.  