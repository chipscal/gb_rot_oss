\documentclass[a4paper, 12pt]{report}
\usepackage[utf8x]{inputenc}
\usepackage[italian]{babel}
%\usepackage[T1]{fontenc}
\usepackage{graphicx}
\usepackage{float}
\usepackage[centertags]{amsmath}
\usepackage{amsfonts}
\usepackage{amssymb}
\usepackage{amsthm}
\usepackage{newlfont}
\usepackage{fancyhdr}
\usepackage{tesisty}
\usepackage{algorithm2e}

%\usepackage{hyperref}

%-------------------------------
% DEFINIZIONE DEGLI ENVIRONMENT
%-------------------------------

\newtheorem{obs}{Osservazione}[section]
\newenvironment{oss}
    {\begin{obs}\begin{normalfont}}
    {\hfill $\square \!\!\!\!\checkmark$ \end{normalfont}\end{obs}}

\newtheorem{pro}{Problema}[chapter]
\newenvironment{prob}
    {\begin{pro}\begin{normalfont}}
    {\hfill $\spadesuit$ \end{normalfont}\end{pro}}

\newtheorem{teor}{Teorema}[section]
\newenvironment{teorema}
    {\begin{teor}\textit }
    {\hfill  \end{teor}}

\newtheorem{defn}{Definizione}[section]
\newenvironment{de}
    {\begin{defn}\begin{normalfont}}
    {\hfill $\clubsuit$ \end{normalfont}\end{defn}}

\newtheorem{prop}{Proprietà}[section]
\newenvironment{pr}
{\begin{prop}\begin{normalfont}}
		{\hfill $\clubsuit$ \end{normalfont}\end{prop}}

%-----------------------------
% CONFIGURAZIONE DELLA PAGINA
%-----------------------------

\hfuzz2pt % Don't bother to report over-full boxes if over-edge is < 2pt

\fancypagestyle{plain}{
\fancyhead{}\renewcommand{\headrulewidth}{0pt} } \pagestyle{fancy}
\renewcommand{\chaptermark}[1]{\markboth{\small CAP. \thechapter \textit{ #1}} {} }
\renewcommand{\sectionmark}[1]{\markright{\small  \thesection \textit{ #1}} {} }
\voffset=-20pt    % distanza tra il limite superiore del foglio e l'intestazione
\headsep=40pt     % distanza  l'intestazione ed il testo del corpo
\hoffset=0 pt     % misura equivalente al margine sinistro
\textheight=620pt % altezza del corpo del testo
\textwidth=435pt  % larghezza del corpo del testo
\footskip=40pt    % distanza tra il testo del corpo ed il pie' di pagina
\fancyhead{}      % cancella qualsiasi impostazione per l'intestazione
\fancyfoot{}      % cancella qualsiasi impostazione per il pie' di pagina
\headwidth=435pt  % larghezza del'intestazione e del pie' di pagina
\fancyhead[R]{\rightmark} \fancyfoot[L]{\leftmark}
\fancyfoot[R]{\thepage}
\renewcommand{\headrulewidth}{0.3pt}   % spessore della linea dell'intestazione
\renewcommand{\footrulewidth}{0.3pt}   % spessore della linea del pi�di pagina

\numberwithin{equation}{section}
\renewcommand{\theequation}{\thesection.\arabic{equation}}




%--------------------------
% MODIFICARE DA QUI IN POI
%--------------------------

\begin{document}

\dedicate{A chi c'è stato, a chi c'è e a chi ci sarà...}

\corso{DELL'AUTOMAZIONE} \titoloTesi{Realizzazione di un software per la stima del posizionamento di un corpo rigido con metodi di visione e validazione attraverso uso motore grafico} \anno{2016/2017}
\relatore{Daniele Carnevale}
 \autore{Luca Calacci}
\correlatore{Corrado Possieri\\ Domenico Cascone}

\baselineskip=25pt

\intestazione

%------------------------------------------------
% INTRODUZIONE E RINGRAZIAMENTI (NON MODIFICARE)
%------------------------------------------------

\fancypagestyle{plain}{
\fancyhead{}\renewcommand{\headrulewidth}{0pt} } \pagestyle{fancy}
\renewcommand{\chaptermark}[1]{\markboth{\small Cap. \thechapter \textit{ #1}} {} }
\renewcommand{\sectionmark}[1]{\markright{\small  \S \thesection \textit{ #1}} {} }
\voffset=-20pt                         % distanza tra il limite superiore del foglio e l'intestazione
\headsep=40pt                          % distanza  l'intestazione ed il testo del corpo
\hoffset=0pt                           % misura equivalente al margine sinistro
\textheight=620pt                      % altezza del corpo del testo
\textwidth=435pt                       % larghezza del corpo del testo
\footskip=40pt                         % distanza tra il testo del corpo ed il pie' di pagina
\fancyhead{}                           % cancella qualsiasi impostazione per l'intestazione
\fancyfoot{}                           % cancella qualsiasi impostazione per il pie' di pagina
\headwidth=435pt                       % larghezza del'intestazione e del pie' di pagina
\fancyhead[R]{\rightmark} \fancyfoot[L]{\leftmark}
\fancyfoot[R]{\thepage}
\renewcommand{\headrulewidth}{0.3pt}   % spessore della linea dell'intestazione
\renewcommand{\footrulewidth}{0.3pt}   % spessore della linea del pi�di pagina

\pagenumbering{Roman} \tableofcontents
\newpage

\pagenumbering{arabic}

\fancyhead[R]{Introduzione} \fancyfoot[L]{Introduzione}
\fancyfoot[R]{\thepage}

\chapter*{Ringraziamenti}
\addcontentsline{toc}{chapter}{Ringraziamenti}



\chapter*{Introduzione}
\addcontentsline{toc}{chapter}{Introduzione}





\fancyhf{} %elimina header/footer vecchi


\fancyhead[R]{\rightmark} \fancyhead[L]{\leftmark}
\fancyfoot[R]{\thepage}





%---------------------
% INCLUSIONE CAPITOLI
%---------------------


\chapter{Strumenti matematici}
\label{chap:mat}

\begin{minipage}{12cm}\textit{In questo capitolo verranno brevemente introdotti gli strumenti che verranno utilizzati in questo lavoro di tesi. Si introdurranno per prima gli strumenti di geometria algebrica quali: ideali, basi di Groebner, teoria dell'eliminazione. Successivamente si procederà ad illustrare la decomposizione SVD.}
\end{minipage}

\vspace*{1cm}

\section{Geometria algebrica}
\label{sec:Geom}

Per comprendere al meglio il lavoro svolto, conviene introdurre alcuni strumenti algebrici fondamentali. Si introdurranno le definizioni algebriche principali e successivamente si introdurrà il concetto di base di Groebner. 

\begin{defn}Si definisce \textbf{monomio} nelle variabili \textit{$x_1, ..., x_n$} un prodotto del tipo $x_1^{\alpha_1} \cdot ... \cdot x_n^{\alpha_n}$ dove gli $\alpha_i$ sono interi non negativi. Si utilizza inoltre la seguente notazione in forma vettoriale \textbf{$x^\alpha$}, dove $x = [x_1 \dots x_n]^T$ e $\alpha = [\alpha_1 \dots \alpha_n]^T$ 	
\end{defn}

\begin{defn}Si definisce \textbf{polinomio} \textit{p} con i coefficienti del campo $\mathbb{K}$ una combinazione $\mathbb{K}$-lineare finita di monomi del tipo
	\begin{center}
		$p = \sum_{\alpha}^{} a_\alpha x^{\alpha}$, $a_\alpha \in \mathbb{K}$
	\end{center}
	si usa indicare con \textbf{termine} l'elemento $\alpha$-esimo della sommatoria.	
\end{defn}

\begin{defn}
	Si usa indicare con $\mathbb{K}[x]$ l'insieme di tutti i polinomi che è possibile generare con monomi nelle variabili $x = [x_1 \dots x_n]^T$ e i coefficienti in $\mathbb{K}$.
\end{defn}

\begin{defn}
	Dato un set di polinomi $p_1(x), ...m p_s(x) \in \mathbb{K}$ si definisce \textbf{varietà affine} il seguente set
	\begin{center}
		$V(p_1(x), ..., p_s(x)) := \{x \in \mathbb{K}^n  \; | \; p_i(x) = 0, i = 1, ..., s \}$
	\end{center} 
\end{defn}

Date due varietà affini $V_1(p_1, ..., p_s)$ e $V_2(q_1, ..., q_t)$ è possibile definire le operazioni di unione e intersezione:
\begin{align}
	\nonumber
	& V_1 \cup V_2 = V(p_1q_1, ...,p_1q_t, p_2q_1, ..., p_2q_t, ..., p_sq_1, ..., p_sq_t), \\
	\nonumber
	& V_1 \cap V_2 = V(p_1, ..., p_s, q_1, ..., q_t).
\end{align}
Si può ora introdurre lo strumento fondamentale utilizzato e alcune sue proprietà.
	
\begin{defn}
	Si definisce \textbf{ideale} \textit{I} un subset di $\mathbb{K}[x]$ che gode delle seguenti proprieta:
	\begin{enumerate}
		\item $0 \in I $,
		\item $f, g \in I \Rightarrow f + g \in I$,
		\item $f \in I, h \in \mathbb{K}[x] \Rightarrow fh \in I$
	\end{enumerate}
\end{defn}
Un modo naturale di definire un ideale è generarlo a partire da un numero finito di polinomi $p_1, ..., p_s$:
\begin{center}
	$\left\langle p_1, ..., p_s \right\rangle := \left\lbrace p \in \mathbb{K}[x] \; | \; p = \sum_{i = 1}^{s} h_ip_i, h_i \in \mathbb{K}[x] \right\rbrace $
\end{center}
Siano $p_1, ..., p_s \in \mathbb{K}[x]$, se si considera un ideale $I = \left\langle p_1, ..., p_s \right\rangle$ esso può essere interpretato nella seguente maniera. Si considerino le seguenti equazioni,
\begin{align}
\nonumber
& p_1 = 0 \\
\nonumber
& \vdots\\
\nonumber
& p_s = 0
\end{align}
se considero $h_1, ..., h_s \in \mathbb{K}[x]$ tali equazioni \textbf{avranno come conseguenza} che
\begin{center}
	$h_1p_1 + ... + h_sp_s = 0$.
\end{center}
Si evince che si può considerare un ideale come l'insieme di tutte le conseguenze generate dal sistema di equazioni omogeneo formato dai polinomi generatori dell'ideale.
\begin{prop}
	Considerati i polinomi $p_1, ..., p_s$ e $q_1, ..., q_r \in \mathbb{K}[x]$ si ha che
	\begin{center}
		$\left\langle p_1, ..., p_s \right\rangle = \left\langle q_1, ..., q_r \right\rangle \Leftrightarrow V(p_1, ..., p_s) = V(q_1, ..., q_r)$
	\end{center}
\end{prop}
Dalla suddetta proprietà si evince che una varietà affine non viene alterata da un cambio di base, cioè essa è definita dall'ideale stesso e non dai polinomi usati come generatori.
\begin{defn}
	Data una varietà affine $V$ si definisce come l'\textbf{ideale di \textit{V}} il set di tutti i polinomi in $\mathbb{K}[x]$ che si annullano su di essa. In formule:
	\begin{center}
		$\mathbf{I}(V) := \left\lbrace p \in \mathbb{K}[x] \; | \; p(x) = 0, \quad \forall x \in V \right\rbrace $
	\end{center}
\end{defn}
In generale vale la seguente proprietà:
\begin{prop}
	Siano $p_1, ..., p_s \in \mathbb{K}[x]$ allora vale che $\left\langle p_1, ..., p_s \right\rangle \subseteq \mathbf{I}(V(p_1, ..., p_s))$. In generale il converso non vale.
\end{prop}
Si può definire infine il concetto di varietà affine di un ideale.
\begin{defn}
	Sia $I$ un ideale in $\mathbb{K}[x]$ il set
	\begin{center}
		$V(I) := \left\lbrace x \in \mathbb{K}^n \; | |; p(x) = 0, \forall p \in I \right\rbrace$
	\end{center}
	si dice \textbf{varietà affine dell'ideale} \textit{I}.
\end{defn}
Vale la seguente proprietà che sarà particolarmente utile alla definizione e determinazione delle preannunciate basi di Groebner.
\begin{prop}
	Sia \textit{I} un ideale in $\mathbb{K}[x]$ e siano $p_1, ..., p_s$ polinomi, allora vale che $V(I) = V(p_1, ..., p_s)$ per ogni base $\left\lbrace p_1, ..., p_s \right\rbrace$ di $I$
\end{prop}

Si consideri un polinomio nella variabile scalare s, del tipo $p = s^n + a_{n-1}s^{n-1} + \cdots + a_1s + a_0$, se considero un ulteriore polinomio $g \in \mathbb{K}[s]$, di grado $k \geq n$ è possibile dimostrare che esso è esprimibile con una relazione del tipo $g = qp + r$, dove $q$ è detto polinomio quoziente e $r$ resto. Inoltre vale che $deg(r) \le deg(g)$. In letteratura esistono diversi algoritmi volti a trovare i polinomio $q$ e $r$, ad esempio l'algoritmo di divisone tra polinomi che estende il concetto di divisione tra scalari.

Un fatto importante da notare è che se si considera un ideale $I = \left\langle g \right\rangle$, dove $g \in \mathbb{K}[x]$, è possibile allora verificare in modo semplice se un polinomio $p \in \mathbb{K}[x]$ appartiene all'ideale $I$ verificando che il resto della divisione di $p$ da $g$ è nullo.

Tale concetto, esteso, porta alla definizione delle suddette basi di Groebner e quindi alla determinazione di una base di ordine ridotto, in senso che verrà presto definito, che definisce un determinato ideale $I$. Al fine di poter estendere tale concetto è necessario poter definire un ordinamento totale tra i polinomi. Nel caso ad una sola variabile, un ordinamento naturale è quello di considerare il grado dei polinomi (l'esponente maggiore con cui compare la variabile del polinomio). Si può estendere tale approccio ma occorre stabilire un ordinamento lessicografico tra i monomi di cui un polinomio si compone. Tale ordinamento può essere definito in diversi modi, nel nostro caso useremo il seguente.

\begin{defn}
	Si considerino due monomi $x^{\alpha}$ e $x^{\beta}$, dove $x = [x_1 \; ... \; x_n]^T$ e $\alpha, \beta \in \mathbb{Z}^n$, si dice che $x^{\alpha}$ e $x^{\beta}$ sono \textbf{in ordine lessicografico}, in simboli $x^{\alpha} \ge_{Lex} x^{\beta}$, con $x_1 \ge x_2 \ge \cdots \ge x_n$ se vale che
	il primo elemento non nullo del vettore $\alpha-\beta$ è maggiore di zero.	
\end{defn}
Si fa notare che il precedente ordinamento è fortemente influenzato dall'ordinamento che si da alle variabili $x_i$, infatti al variare di tale ordinamento (e corrispettiva variazione dell'ordinamento sui vettori $\alpha$ e $\beta$) si ottengono ordinamenti diversi.

\begin{defn}
	Dato un polinomio $p \in \mathbb{K}[x]$ se si definisce un ordine lessicografico e si riscrive tale polinomio in modo ordinato, $ p = a_1x^{\alpha_1} + a_2x^{\alpha_2} + \cdots $,  si definiscono il \textbf{leading monomial} $LM(p) := x^{\alpha_1}$, il \textbf{leading coefficient} $LC(p) := a_1$ e il \textbf{leading terms} $LT(p) := a_1x^{\alpha_1}$
\end{defn}
Si può estendere quindi il concetto di divisione tra polinomi, al fine di ottenere un algoritmo multi-variabile in grado di dividere un polinomio per un set di polinomi. In altre parole dato un set di polinomi $p_1, ..., p_s$ e un polinomio $p$ appartenenti a $\mathbb{K}[x]$ è possibile trovare $q_1, ... , q_s, r \in \mathbb{K}[x]$ tali che $p = q_1p_1 + q_1p_2 + \cdots + q_sp_s + r$ e vale che o $r = 0$ oppure $r = c_1x^{\beta_1} + c_2x^{\beta_2} + \cdots $ e nessun $c_ix^{\beta_i}$ è divisibile per nessun $LT(p_j)$ con $i = 1,2,..., j = 1, 2,..., s$.

\begin{obs}
	Si può quindi facilmente capire che dato un ideale $I = \left\langle p_1, ..., p_s \right\rangle, p_i \in \mathbb{K}[x]$ e un ulteriore polinomio $p \in \mathbb{K}[x]$, quest'ultimo diviso per il set suddetto, si otterrà un resto nullo se e solo $ p \in I$. In altre parole si è trovato uno strumento per verificare se un polinomio è conseguenza di un ideale.
\end{obs}

Si riporta in seguito un algoritmo, chiamato \textbf{Multivariate division algoritmh}, atto a tale scopo. \\

\begin{algorithm}[H]
	\SetAlgoLined
	\KwData{una tupla ordinata $p_1, ..., p_s$ di polinomi $\in \mathbb{K}[x]$ e un polinomio $p \in \mathbb{K}[x]$}
	\KwResult{$q_1, ..., q_s, r \in \mathbb{K}[x]$}
	set $q_1 := 0, ..., q_s := 0$\;
	set r := 0\;
	set f := p\;
	\While{f $\neq$ 0}{
		i = 1; divisionOccurred = false\;
		\While{$i \leq s$ AND divisionOccurred == False}{
			\eIf{LT($p_i$) divide LT(f)}{
				$q_i = q_i + \frac{LT(f)}{LT(p_i)}$\;
				$f = f - \frac{LT(f)}{LT(p_i)}p_i$\;
				divisionOccurred = True\;
			}{
				$i = i + 1$\;
			}
		}
		\If{divisionOccurred == False}{
			$r = r + LT(f) $\;
			$f = f - LT(f) $\;
		}
	}
	\caption{Multivariate division algoritmh}
\end{algorithm}

\begin{prob}
	\label{prob:mvda}
	Il risultato del precedente algoritmo è fortemente dipendente dall'ordinamento dato ai polinomi $p_i$. Si vogliono trovare, se esistono, strumenti tali da rendere il calcolo del resto di tale divisione indipendente dall'ordinamento dato al set dei polinomi $p_i$.
\end{prob}

Il problema \ref{prob:mvda} costituisce la motivazione per introdurre le preannunciate basi di Groebner (ridotte), le quali, come si vedrà, permettono di risolvere tale problema.

\begin{defn}
	Si consideri l'ideale $I \subset \mathbb{K}[x]$ non vuoto, si fissi un ordinamento per i monomi e si definisca l'insieme
	\begin{center}
		$LT(I) := \left\lbrace cx^{\alpha} monomio \; | \; \exists f \in I \; \; \; t.c. \; \;  LT(f) = cx^{\alpha} \right\rbrace $
	\end{center} 
	e l'ideale dei leading terms generato da tale insieme $\left\langle LT(I) \right\rangle $.
	Un subset finito $G := \left\langle g_1, ..., g_m\right\rangle \subset I$ si dice una \textbf{base di Groebner} per l'ideale $I$ se vale
	\begin{center}
		$\left\langle LT(g_1), ..., LT(g_m) \right\rangle = \left\langle LT(I) \right\rangle$
	\end{center}
\end{defn}

\begin{prop}
	Sia $\left\langle \O \right\rangle := \{0\}$ il subset vuoto e sia esso per convenzione una base di Groebner per l'ideale vuoto, allora per ogni ideale esiste (non necessariamente unica) una base di Groebner che è una base per tale ideale.
\end{prop}

\begin{defn}
	Un polinomio $r \in \mathbb{K}[x]$ si dice \textbf{ridotto} rispetto ad un set $\left\lbrace g_1, ..., g_m \right\rbrace$, $g_i \in \mathbb{K}[x]$, se esso è esprimibile come una combinazione $\mathbb{K}$-lineare di termini nessuno dei quali è divisibile per nessuno dei $LT(g_1), ..., LT(g_m)$.
\end{defn}

\begin{defn}
	Una base di Groebner $\left\lbrace g_1, ..., g_m \right\rbrace$ si dice \textbf{ridotta} se vale che:
	\begin{enumerate}
		\item $LC(g_i) = 1, i = 1, ..., m$,
		\item $g_i$ è ridotto rispetto al set $\left\lbrace g_1, ...g_{i-1},g_{i+1}, ..., g_m \right\rbrace, i = 1, ..., m$
	\end{enumerate}
\end{defn}

\begin{prop}
	Per ogni ideale $I \subset \mathbb{K}[x]$ esiste un unica base di Groebner ridotta.
\end{prop}

\begin{prop}
	Sia $p \in \mathbb{K}[x]$ e $G$ una base di Groebner per l'ideale $I \subset \mathbb{K}[x]$. Allora il resto della divisione di $p$ per il set $G$ è unico indipendentemente dall'ordine dei polinomi $g_i$.
\end{prop}
Risulta abbastanza chiaro che l'uso delle basi di Groebner per definire un ideale permette di risolvere il problema \ref{prob:mvda}; infatti un polinomio $p \in I$ se e solo se vale che il resto della divisione tra $p$ e la base di Groebner di $I$ è nullo.
Si pone a questo punto il problema della determinazione di una base di Groebner dato un ideale I. A tal fine si può utilizzare l'algoritmo di Buchberger.

\begin{algorithm}[H]
	\SetAlgoLined
	\KwData{un set $p_1, ..., p_s$ di polinomi $\in \mathbb{K}[x]$}
	\KwResult{una base di Groebner $G$ per $\left\langle p_1, ..., p_s \right\rangle $}
	$G = \left\lbrace p_1, ..., p_s \right\rbrace $\;
	\Repeat{$G == $\^{G}}{
		\^{G} = G\;
		\ForEach{$p, q \in$ \^{G}, $p \neq q$}{
			$x^{\alpha} = LM(p)$; $x^{\beta} = LM(q)$\;
			$\gamma = (\gamma_1, ..., \gamma_n)$; $\gamma_i = max(\alpha_i, \beta_i)$\;
			$s = \frac{x^{\gamma}}{LT(p)}p - \frac{x^{\gamma}}{LT(q)}q$\;
			r := resto della divisione tra $s$ e il set \^{G}\;
			\If{$r \neq 0$}{
				$G = G \cup {r}$\;
			}
		}
	}
	
	\caption{Buchberger's algoritmh}
\end{algorithm}

\medskip
Si noti che il suddetto algoritmo non fornisce in genere una base di Groebner ridotta, infatti mantiene in G i polinomi $p_i$. 
Si fa notare che lo scopo del polinomio $s$ è quello di permettere cancellazioni tra i leading terms dei polinomi $p$ e $q$, ci si riferisce ad esso come il polinomio-S, in simboli $S(p,q)$. Infine la condizione di terminazione è corretta perché è possibile dimostrare che un set di polinomi $G = \{g_1, ..., g_m\}$ è una base di Groebner per un certo ideale $I$ se e solo se per ogni $i,j = 1,..., m , \; i\neq j$ il resto della divisione di $S(g_i, g_j)$ per il set $G$ è nullo (Buchberger's criterion). Si fa presto a capire che siccome per ogni coppia (p,q) si deve effettuare una divisione con tutti i polinomi in \^{G} ed inoltre per ogni iterazione si aggiungerà un polinomio a tale set, l'algoritmo risulta pesante dal punto di vista computazionale. Infatti, si può dimostrare che l'algoritmo è EXPSPACE-complesso, cioè con tempo di calcolo e occupazione di memoria esponenziali. In genere però esso risulta applicabile sui computer moderni.
Nel seguito della trattazione si utilizzerà il software Macaulay2 per il calcolo delle basi di Groebner di un dato ideale, il quale implementa proprio il suddetto algoritmo.


% --------- elimination theory
\subsubsection{Teoria dell'eliminazione}
Durante gli studi algebrici sicuramente ci si può essere imbattuti in un noto algoritmo detto: \textbf{Eliminazione di Gauss}. Quest'ultimo permette, dato un sistema lineare di equazioni - rappresentabile attraverso una matrice, attraverso operazioni elementari su sopraddetta matrice (operazioni che non ne alterano il rango), di ottenere una rappresentazione, in ipotesi di rango riga pieno, detta (quasi-)triangolare superiore. In tal modo si ottiene un sistema di equazioni in cui le ultime equazioni non dipendono da un certo numero delle prime variabili. Selezionando un certo numero delle ultime equazioni, è possibile ottenere un sotto-sistema in cui le prime variabili non compaiono (sono state eliminate). 
Tale approccio viene utilizzato per risolvere in maniera più semplice un sistema di equazioni, in ipotesi che possa essere fatto.

Tale concetto può essere esteso al caso polinomiale, non lineare, multi-variabile.

\begin{defn}
	Un set di polinomi $p_1, ..., p_m$, $p_i \in \mathbb{K}[x]$, si dice linearmente dipendente se esistono $c_1, ... c_m$ costanti non tutte nulle tale che $c_1p_1 + c_2p_2 + \, ... \, + c_mp_m = 0$. Altrimenti si dice \textbf{linearmente indipendente}.
	
	Un set di polinomi $p_1, ..., p_m$, $p_i \in \mathbb{K}[x]$, si dice algebricamente dipendente se esiste $q \in \mathbb{K}[y_1, \, ... \, , y_m]$ polinomio non nullo, tale che  $q(p_1, \, ..\, p_m) = 0 \in \mathbb{K}[x]$ (è il polinomio nullo). Altrimenti si dice \textbf{algebricamente indipendente}.
\end{defn}

Dato un ordinamento lessicografico $x_1 \ge_{Lex} x_2 \ge_{Lex} ... \ge_{Lex} x_n$ può essere possibile eliminare le prime $n-1$ variabili. Si dice infatti che tale ordine \textbf{elimina} $x_1, \, ... \, , x_{n-1}$.

\begin{defn}
	Dato un ordinamento lessicografico $x_1 \ge_{Lex} x_2 \ge_{Lex} ... \ge_{Lex} x_n$ e un ideale $I \subset \mathbb{K}[x]$ allora l'ideale
	\begin{center}
		$I_l = I \cap \mathbb{K}[x_{l+1}, \, ... \, x_{n}]$
	\end{center}
	si dice l'ideale di eliminazione di ordine $l$.
\end{defn} 
L'ideale di eliminazione di ordine $l$ può essere visto come l'insieme di tutte le conseguenze di un ideale nelle quali le prime $l$ variabili non compaiono.

Si può introdurre infine il teorema generale di eliminazione.
\begin{teor}
	\label{teo:elimination}
	Sia dato un ideale $I \subset \mathbb{K}[x_a, x_b]$ con $x_a \in \mathbb{R}^n_a, x_b \in \mathbb{R}^n_b$, $ n_a, n_b > 0$, $n = n_a + n_b$, e un ordinamento lessicografico $x_a \ge_{Lex} x_b$. Inoltre sia $G$ una base di Groebner ridotta per l'ideale $I$, calcolata in accordo a tale ordinamento. Allora vale che una base di Groebner per l'ideale di eliminazione di ordine $n_a$ è 
	\begin{center}
		$G_{n_a} = G \cap \mathbb{K}[x_b]$.
	\end{center}
\end{teor}

Mediante l'utilizzo delle basi di Groebner e il precedente teorema è possibile quindi estendere il concetto di eliminazione, per applicarlo a sistemi di equazioni omogenei multi-variabile polinomiali non lineari. In modo di ottenere delle conseguenze, sistemi omogenei di dimensione ridotta dello stesso tipo, che non contengano alcune variabili. Tale strumento può essere utilizzato per ottenere soluzioni analitiche ai problemi studiati. 

\section{Decomposizione ai valori singolari}
\label{sec:SVD}

In algebra lineare, la decomposizione ai valori singolari, detta anche SVD (dall'acronimo inglese Singular Value Decomposition), è una particolare fattorizzazione di una matrice basata sull'uso di autovalori e autovettori. 

\begin{defn}
	Data una matrice $M \in \mathbb{C}^{m \times n}$, si definisce decomposizione ai valori singolari o SVD la seguente decomposizione:
	\begin{center}
		$M = U \Sigma V^*$
	\end{center}
	dove $U$ è una matrice unitaria di dimensioni  $m\times m$, $\Sigma$ è una matrice diagonale rettangolare di dimensioni $m\times n$ e $V^*$ è la trasposta coniugata di una matrice unitaria $V$ di dimensioni $n\times n$.
	Si definiscono inoltre:
	\begin{enumerate}
		\item gli elementi di $\Sigma$ valori singolari della matrice $M$,
		\item le colonne di $U$ vettori singolari sinistri della matrice $M$,
		\item le righe di $V^*$ covettori singolari destri  della matrice $M$.		
	\end{enumerate}
\end{defn}

\begin{prop}
	Data una decomposizione SVD valgono le seguenti proprietà:
	\begin{enumerate}
		\item se $ m \le n$, allora $\Sigma = [\Sigma_0 \, \, 0]$, dove $\Sigma_0 = diag \{\sigma_1, ..., \sigma_m \}, \sigma_i \in \mathbb{R}, \sigma_i \ge 0$,
		\item se $ m > n$, allora $\Sigma = [\Sigma_0^T \, \, 0^T]^T$, dove $\Sigma_0 = diag \{\sigma_1, ..., \sigma_n \}, \sigma_i \in \mathbb{R}, \sigma_i \ge 0$,	 
		\item i valori singolari $\sigma_i$ non nulli di $M$ sono le radici quadrate degli autovalori della matrice $MM^*$ e $M^*M$,
		\item i vettori singolari di sinistra di $M$ sono gli autovettori di $MM^{*}$,
		\item i covettori singolari di destra di $M$ sono gli autovettori di $M^{*}M$ trasposti.
	\end{enumerate}
\end{prop}

Per il calcolo della decomposizione SVD sebbene è possibile procedere usando la definizione data, questo è sconsigliato perché inefficiente. Negli ultimi anni sono stati sviluppati diversi algoritmi atti allo scopo. Nello specifico si cita l'algoritmo Golub/Kahan utilizzato nelle recenti librerie matematiche come LAPACK, BLAS e CUBLAS.

In particolare si fa notare che questi algoritmi sono particolarmente efficienti quando la matrice da decomporre è quadrata.

La fattorizzazione SVD ha numerose applicazioni pratiche, infatti:
\begin{enumerate}
	\item il rango di una matrice è pari al numero di valori singolari maggiori di zero nella sua decomposizione SVD.
	\item può essere usata per il calcolo della matrice pseudo-inversa destra (o sinistra). Infatti sia $M \in \mathbb{C}^{m \times n}, rank(M) = m$ allora la sua decomposizione sarà $M = U [ \Sigma_0 \,\, 0 ] V^*, \Sigma_0 > 0, UU^* = I, VV^* = I$, allora una pseudo inversa destra per $M$ può essere calcolata come:
	\begin{center}
		$M^{R} = V 
		\begin{bmatrix}
			\Sigma_0^{-1} \\
			Z
		\end{bmatrix} U^*$
	\end{center} 
	dove $Z$ è una qualsiasi matrice che può essere anche nulla e $\Sigma_0^{-1}$ è sempre diagonale e cui elementi saranno del tipo $1/\sigma_i$.
	\item come vedremo nel paragrafo \ref{sec:kabsch} è particolarmente utile per risolvere problemi ai minimi quadratici.
\end{enumerate}
\chapter{Visual Observer}
\label{chap:visualObs}

\begin{minipage}{12cm}\textit{blabla..}
\end{minipage}

\vspace*{1cm}



\section{Motivazioni}
\label{sec:motivi}


\section{Approccio al problema}
\label{sec:approccio}
\chapter{Stima della matrice di trasformazione}
\label{chap:stima}

\begin{minipage}{12cm}\textit{blabla..}
\end{minipage}

\vspace*{1cm}

%qua descrivo il problema...


\section{Il metodo di Groebner}
\label{sec:groeb}



\section{Il metodo di Kabsch}
\label{sec:kabsh}
\chapter{Visione artificiale}
\label{chap:visione}

\begin{minipage}{12cm}\textit{In questo capitolo verrà trattato il problema della generazione dei set di coppie di punti omologhi usando tecniche di visione. In particolare si utilizzeranno tecniche di Feature Detection e stereoscopia.}
\end{minipage}

\vspace*{1cm}

Nel capitolo \ref{chap:visualObs} si è accennato agli strumenti che sarebbero stati usati al fine della generazione dei suddetti set. In questo capitolo si procederà ad illustrarne il funzionamento con maggior dettaglio e si fornirà una possibile soluzione al problema. Conviene definire per primo il problema che si vuole affrontare.

\begin{prob}
	\label{prob:vis:gensets}
	Sia un osservatore mobile dotato di almeno due fotocamere. Si supponga ancora capace di "catturare", attraverso le suddette fotocamere, coppie di foto in diversi istanti (anche regolari). Allora: 
	\begin{enumerate}
		\item è possibile, utilizzando le informazioni fornite dalle fotocamere, calcolare le coordinate spaziali (in $\mathbb{R}^3$) di un punto del mondo (o più di uno) nel sistema di riferimento solidale all'osservatore?
		\item dato un punto del mondo (o più di uno) per il quale si è riusciti a calcolarne le coordinate spaziali, è possibile individuare lo stesso punto, in un altro istante del tempo, se l'osservatore si è mosso e se lo stesso è ancora visibile?  
	\end{enumerate}
\end{prob}

\section{Feature Detection e Matching}
\label{sec:feature}
Per \textbf{Feature Detection} o \textbf{individuazione dei punti chiave} in italiano, si intende un particolare settore facente parte della branca della visione artificiale, il cui scopo è quello di per l'appunto l'individuazione, la descrizione e il confronto delle Feature. Ma che cosa è una Feature?
In generale risulta difficile fornire una definizione chiara di cosa è una \textbf{Feature} o \textbf{Punto chiave}, può dipendere ad esempio dall'algoritmo che si utilizza o da scelte implementative.


\begin{figure}[h]
	\centering
	\includegraphics[width=420pt]{imgs/feature_simple.png}
	\caption{Individuazione delle Features.}
	\label{vis:feature:detect}
\end{figure} 

Si faccia ad esempio riferimento alla figura \ref{vis:feature:detect}, si supponga di dover individuare una buon punto chiave, il quale possa essere facilmente riconosciuto e riposizionato. Si consideri per prima la porzione di immagine evidenziata dal riquadro blu; è chiaro che qualora si dovesse scegliere dove posizionare tale blocco, si avrebbero un numero elevatissimo di possibilità ed in nessun caso si potrebbe avere al certezza di un buon posizionamento. Si consideri allora il blocco nero, in questo caso il numero delle possibilità è sicuramente inferiore, infatti esso può essere posizionato solo nel lato inferiore del rettangolo verde in figura. Infine si consideri il blocco rosso, è chiaro che quello in figura è l'unico punto in cui esso può essere posizionato.
L'esempio precedente fornisce un importantissimo spunto ri riflessione su cosa può essere una Feature e cosa no. Infatti sicuramente una porzione di immagine uniforme non può essere un punto chiave; se invece si considerano dei contorni di un oggetto la situazione migliora. In particolare se si considera il contorno di un oggetto e se ne sceglie un punto che abbia una forma o pattern particolare, come ad esempio uno spigolo di un oggetto, risulta molto più facile un eventuale riconoscimento. 

Ovviamente l'esempio precedente risulta estremizzato. Infatti, in un caso reale, le Feature possono essere ruotate e/o scalate e questo complica il problema. Al contempo però, il mondo reale risulta più dettagliato e risulta più difficile avere uno stesso pattern in punti diversi.

Si è fornita, quanto meno in modo informale, una definizione di Feature. Nei prossimi paragrafi con ordine si illustreranno tre algoritmi: il primo adibito all'individuazione delle features, il secondo viene impiegato per la generazione dei descrittori (che come si vedrà sono utilizzati per distinguere i punti chiave in base alle informazioni fornite dall'immagine) ed il terzo è un matcher. 

%\subsection{Definizioni e problema}
%\label{sec:det:def}
\subsection{L'algoritmo FAST}
\label{sec:det:fast}
L'algoritmo FAST (Feature from Accelerated Segment Test) è un algoritmo adibito all'estrazione dei punti chiave da un immagine, sviluppato originariamente da Edward Rosten e Tom Drummond nel 2006 \cite{bib2}. Esistono numerosi algoritmi adibiti a questo scopo, ed alcuni riescono anche ad ottenere risultati più accurati. In accordo con il nome dell'algoritmo però quest'ultimo risulta particolarmente veloce e in generale fornisce risultati molto buoni. In accordo con le specifiche di esecuzione realtime già accennate, pertanto la scelta è ricaduta su quest'ultimo. Inoltre l'algoritmo FAST è privo di licenze a pagamento e può essere liberamente impiegato in progetti sia di ricerca che commerciali. Di seguito se ne illustra il funzionamento.
\newline \newline
Si consideri un pixel $p$, sia $I_p$ la sua intensità e siano fissate una soglia di funzionamento $t$ e un intero $n$. Si vuole verificare se il pixel $p$ può essere considerato una Feature oppure no. 

\begin{figure}[h]
	\centering
	\includegraphics[width=420pt]{imgs/fast_speedtest.jpg}
	\caption{Fast Segment test.}
	\label{vis:feature:Fast}
\end{figure} 
  
 
In accordo con quanto detto nel precedente paragrafo, uno spigolo/angolo può essere considerato una buon punto chiave. Il \textbf{test} ricerca questo tipo di punti e viene effettuato nel modo seguente:
\begin{enumerate}
	\item Si consideri un cerchio di 16 pixel intorno a $p$, evidenziato in figura \ref{vis:feature:detect} dai pixel riquadrati.
	\item Il punto $p$ è uno spigolo se vale che un numero $n$ di pixel appartenenti al suddetto cerchio hanno intensità o tutti maggiore di $I_p + t$ o tutti minore di $I_p - t$.
\end{enumerate}

Osservando sempre la figura \ref{vis:feature:detect} si intuisce la motivazione del test. Infatti uno spigolo, indipendentemente da come è ruotato, avrà una forma convessa. Si intuisce che la buona riuscita del test dipende dalla scelta dei parametri $t, n$, tipicamente si sceglie $n = 12$, $t$ dipende dal tipo di immagini utilizzate.

	Nella pratica si inserisce un \textbf{test veloce} per effettuare una prima scrematura dei punti. Con riferimento alla figura \ref{vis:feature:detect} si controllano solo i pixel 1, 9, 5 e 13. Si verifica se i pixel 1 e 9 hanno una differenza assoluta di intensità rispetto a $p$ maggiore della soglia $t$, in caso si analizzano anche i pixel 5 e 13. Se $p$ è uno spigolo deve valere allora che 3 di tali pixel devono essere più intensi di $p$ e uno meno intenso, dove l'intensità e verificata con al soglia $t$ ovviamente.

Si può notare che se la condizione del test veloce non è verificata allora non c'è modo che il test completo possa avere un buon esito, perché non possono esistere n pixel contigui più intensi o meno intensi di $p$. In generale dei falsi positivi possono superare il test veloce e quindi il testo completo deve essere impiegato per ottenere un risultato corretto.

L'algoritmo allo stato attuale però presenta alcune criticità:
\begin{enumerate}
	\item Per $n < 12$ il test veloce non può essere generalizzato perché un pixel $p$ può superare il test completo anche se solo due dei pixel selezionati sono meno intensi e due più intensi.
	\item L'efficienza del test dipende dalla scelta e dall'ordinamento dei pixel selezionati per il test. In genere però i punti per il successo del test dipende da questi fattori non sono buone feature. 
	\item Molte feature vengono individuate una adiacente all'altra.
\end{enumerate} 

Per risolvere il terzo problema si introduce un ulteriore passo detto \textbf{non-maximal suppression}:
\begin{enumerate}
	\item Si calcola una funzione valore $V$ per tutti i punti chiave individuati. La funzione $V$ per il punto chiave $p$ è pari alla somma della differenza assoluta tra l'intensità di $p$ e i 16 pixel considerati intorno allo stesso durante il test.
	\item Se i punti chiave $p_1$ e $p_2$ sono adiacenti, ovvero hanno una distanza in pixel minore di una soglia $d$, si scarta quello che ha la funzione valore più bassa.
\end{enumerate}
Il precedente test è pensato in modo che, in caso di necessità, vengono scartati i punti chiave meno "marcati" e conseguentemente meno distinguibili. 

L'ultimo passo che può essere impiegato per aumentare le prestazioni del rilevatore FAST è quello di impiegare una \textbf{rete neurale}. Essa può essere addestrata utilizzando i risultati forniti dello stesso FAST durante il funzionamento al fine di sostituire e combinare il test veloce e quello completo. A patto che il set di addestramento sia abbastanza vario e comprensivo di rumore, un addestramento supervisionato può ottenere un buon grado di efficacia ed efficienza. 
% scrivere sta roba%

Si conclude la presentazione dell'algoritmo con alcune riflessioni. FAST è un algoritmo di tipo euristico, come molti altri, ma nel caso generale risulta molto efficace. Per quanto riguarda il test in essere esso ha un basso impatto computazionale, inoltre il test può essere eseguito in parallelo per molti pixel alla volta. Come vedremo esso si presta ad una facile e efficiente implementazione su GPU.

\subsection{L'algoritmo BRIEF}
\label{sec:det:brief}

\subsection{Brute Force matching}
\label{sec:det:bmmatch}


\section{Stereoscopia}
\label{sec:stereo}


%\subsection{Definizioni e problema}
%\label{sec:stereo:def}


\subsection{Modello matematico del sensore fotografico}
\label{sec:stereo:modello}


\subsection{Ricostruzione 3D}
\label{sec:stereo:ric3d}


\section{Soluzione al problema}
\label{sec:vision:solution}



\chapter{Implementazione e test}
\label{chap:implTest}

\begin{minipage}{12cm}\textit{In questo capitolo si illustreranno gli aspetti tecnici relativi alla realizzazione del software: la scelta delle librerie e dell'ambiente di sviluppo, i problemi riscontrati e le soluzioni intraprese. Si procederà inoltre alla validazione del sistema e la valutazione delle prestazioni.}
\end{minipage}

\vspace*{1cm}

\section{Lo sviluppo}
\label{sec:sviluppo}
Nei precedenti capitoli si è discusso principalmente degli strumenti teorici che, come spesso segnalato, sono stati selezionati o sintetizzati con in mente il requisito di efficienza imprescindibile in un contesto di controllo nel quale ci si pone. Sono stati selezionati perciò gli algoritmi migliori rispetto ad un rapporto efficacia/costo computazionale. Inoltre, si sono scelti quelli meglio parallelizzabili su architetture vettorizzate o dotate di co-processori grafici. Nel seguito verrà descritto l'aspetto implementativo del software Visual Observer. Si descriveranno in particolare gli strumenti di sviluppo, le librerie usate e l'architettura software implementata. 

\subsection{Ambiente di sviluppo e target}
\label{sec:dev:ambiente}

Prima di poter scrivere anche una sola linea di codice, si deve aver ben chiaro l'obbiettivo di sviluppo che ci si pone al fine di poter valutare quale può essere l'ambiente di sviluppo, il linguaggio e le librerie da utilizzare.  Conviene allora specificare e definire i requisiti di progetto:
\begin{itemize}
	\item \textbf{Architettura target:}  x86\_64, ARM,
	\item \textbf{Sistema Operativo target:}  Windows, GNU/Linux,
	\item \textbf{Specifiche hardware minime:}
	\begin{itemize}
		\item CPU: x86\_64 o ARM,
		\item GPU: nessuna o generica con supporto OpenCL,
	\end{itemize} 
	\item \textbf{Specifiche hardware consigliate:}
		\begin{itemize}
			\item CPU: x86\_64 o ARM con supporto OpenCL o SIMD/AVX,
			\item GPU: coprocessore grafico NVidia ad alte prestazioni con supporto CUDA Shared Model $\ge$ 2.1,
		\end{itemize}			 
\end{itemize}

La scelta delle architetture e dei sistemi operativi non è casuale, ovviamente. Infatti, il supporto x86\_64 permette di poter sviluppare e testare il codice su un normale PC in ambiente Windows o Linux. D'altro canto il software è pensato per essere utilizzato a bordo di un sistema di controllo di un robot e tipicamente quest'ultimi hanno processori ARM su sistema Linux.

Spesso si è fatto riferimento nei capitoli precedenti ad un requisito di esecuzione real-time. Formalmente esso è difficile da definire, si può intendere però nel seguente modo: \textit{si fissi una piattaforma hardware abbastanza potente, si richiede che il tempo di esecuzione del software di stima sia tale da rendere la stessa disponibile in tempo utile.} Ad esempio, su sistemi moderni si potrebbe richiedere che sia in grado si fornire una stima con frequenza 20 Hz o superiore.
 
Si può comprendere allora la specifica richiesta della presenza di un processore vettorizzato o un coprocessore grafico con supporto a OpenCL o CUDA. Infatti, in questi casi è possibile ridurre di moltissimo i tempi di esecuzione grazie alla parallelizzazione dei calcoli.

Alla luce di quanto su detto, la scelta del linguaggio di sviluppo è ricaduta sul \textbf{C++}. Infatti esso permette di ottenere un codice macchina particolarmente ottimizzato e al contempo di mantenere la portabilità dello stesso. Il codice è stato sviluppato utilizzando il sistema operativo \textbf{Windows} e l'ambiente di sviluppo \textbf{Visual Studio Community 2015}. Tale scelta è motivata da diversi fattori:
\begin{enumerate}
	\item Allo stato attuale, Visual Studio permette di sviluppare e compilare codice, in modo semplice, per tutti i sistemi operativi e le architetture su citati.
	\item In ambiente Linux non si sarebbe potuto sviluppare per Windows.
	\item Rende lo sviluppo più semplice e veloce, grazie anche a numerosi strumenti di debug e profiling messi a disposizione.
	\item Per scopi accademici (e piccoli progetti commerciali) è distribuito gratuitamente.
\end{enumerate}

Nel seguito si presenta una breve descrizione delle librerie software utilizzate: OpenCV e CUDA. Non si farà riferimento a OpenCL perché non direttamente utilizzata. Infatti, grazie all'uso di OpenCV l'utilizzo della stessa o di un altro sistema di vettorizzazione è del tutto trasparente. Come sarà chiaro più avanti nel paragrafo dedicato al testing, l'attivazione o meno di quest'ultima però darà luogo a prestazioni nettamente superiori. Informalmente, OpenCL è un architettura hardware e software per il calcolo parallelo. Essa permette di utilizzare sia CPU che GPU (anche Nvidia), talvolta in contemporanea, per effettuare calcoli generici.

\subsection{OpenCV}
\label{sec:tools:opencv}

OpenCV, acronimo in lingua inglese di \textbf{Open} source \textbf{C}omputer \textbf{V}ision library, è una libreria software multi piattaforma per la visione artificiale. Essa nella sua prima versione risale al lontano 2000 ed è stata originariamente sviluppata ad opera di Intel. Successivamente il progetto è stato preso in carico dalla comunità open source ed è mantenuto da un numero elevato di appassionati e sviluppatori. Oggi giorno, è considerata lo standard di fatto utilizzato per progetti comprendenti la visione artificiale in progetti sia accademici che commerciali. Essa comprende molteplici ambiti tra cui: calcolo matriciale e strumenti di algebra lineare, riconoscimento delle feature, stereoscopia e filtraggio delle immagini. Di giorno in giorno gli strumenti presenti vengono ottimizzati e di nuovi ne vengono aggiunti. 
La libreria è scritta principalmente in codice C++, fortemente ottimizzato per sfruttare le particolarità della piattaforma target: vettorizzazione SIMD e AVX, openCL, CUDA. Risulta perciò un valido strumento in progetti di applicazioni real-time. Esistono poi numerosi wrapper che permettono il suo impiego in altri linguaggi tra cui: Python, Java, C\#. Inoltre è disponibile per praticamente tutti i maggiori sistemi operativi: Windows, Linux, Android, IOs e OsX e per le maggiori architetture: x86, x86\_64, ARM.
Infine si segnala che è rilasciata sotto licenza BSD perciò può essere liberamente impiagata in progetti sia accademici che commerciali.

\subsection{CUDA}
\label{sec:tools:cuda}
CUDA (acronimo di Compute Unified Device Architecture) è un'architettura hardware per l'elaborazione parallela creata da NVIDIA. Essa, unitamente all'ambiente di sviluppo CUDA, permette scrivere applicazioni capaci di eseguire calcolo parallelo sulle GPU Nvidia. I linguaggi di programmazione disponibili nell'ambiente di sviluppo per CUDA, sono estensioni dei linguaggi più diffusi per scrivere programmi. Il principale è "CUDA-C" (C con estensioni NVIDIA), altri sono estensioni di Python, Fortran, Java e MATLAB.

Diversamente dalle CPU, le GPU hanno un'architettura parallela con diversi core, ognuno capace di eseguire centinaia di processi simultaneamente, se un'applicazione è adatta per quel tipo di architettura la GPU può offrire grandi prestazioni e benefici. Questo approccio di risoluzione dei problemi è noto come GPGPU (General Purpose GPU). In generale un ottimo contesto di esecuzione per questo tipo di dispositivi è quando vengono avviati moltissimi thread con lo stesso flusso di esecuzione. In realtà infatti le GPU estremizzano il concetto di vettorizzazione. Si hanno a disposizione un certo numero di Shared Processor, capaci di eseguire un gruppo di thread alla volta detto Warp. Per poter eseguire un Warp in modo parallelo deve accadere che tutti i thread che lo compongono eseguano le stesse operazioni, ovviamente su indirizzi di memoria differenti. Inoltre per rendere gli accessi in memoria efficienti e ridurre (anche drasticamente) i tempi di calcolo i thread devono poter accedere alla memoria in modo adatto. Infatti, le GPU gestiscono la memoria in modo opposto rispetto alle CPU, cioè locazioni di memoria vicine sono disposte in diversi banchi di memoria. Il motivo di tale scelta risiede nel fatto che le richieste in memoria sono solitamente molto lente e si cerca di parallelizzare la lettura in memoria al fine di rendere i dati disponibili simultaneamente per tutti i thread di un Warp. 

Alla luce di ciò, è chiaro che non tutte le applicazioni possono essere parallelizzate su coprocessore grafico ed è inoltre chiaro perché un forte accento si è posto sul trovare algoritmi parallelizzabili. Ad esempio questo è il caso del comparatore a forza bruta, presentato nel paragrafo \ref{sec:det:bmmatch}, che seppur utilizza un approccio poco intelligente risulta molto veloce.

\subsection{L'architettura software}
\label{sec:dev:arch}
Infine, per concludere il paragrafo, si presenta l'architettura software sviluppata. Essa è pensata per essere completamente modulare ed intercambiabile. I blocchi funzionali contraddistinti con la lettera "I" maiuscola sono delle interfacce e diverse implementazioni ne sono fornite.
\begin{figure}[h!]
	\centering
	\includegraphics[width=400pt]{imgs/arch.png}
	\caption{Rappresentazione architettura software Visual Observer.}
	\label{vis:impl:arch}
\end{figure} 

L'architettura inoltre prevede che essi possano essere sostituiti da future implementazioni. Con riferimento alla figura \ref{vis:impl:arch} è possibile verificare che il sistema Visual Observer si compone di tre principali blocchi funzionali: 
\begin{enumerate}
	\item \textbf{Camera Manager}: ha lo scopo di gestire le camere e astrarre l'acquisizione dell'input visuale verso il Visual Observer. Esso può inoltre gestire più camere anche di diverso tipo. Le camere possono essere registrate attraverso l'interfaccia \textbf{ICamera} e un identificativo univoco. In tal modo è possibile richiedere al manager di acquisire un immagine da un determinato sensore. Si è prevista inoltre la possibilità di richiedere l'input sincrono da parte di due sensori grafici allo scopo di ottenere una coppia di immagini stereo catturate nello stesso istante.
	L'interfaccia ICamera astrae il manager dal funzionamento del sensore stesso e permette quindi l'utilizzo di sensori di diverso tipo senza dover modificare il codice a monte. Tale interfaccia espone diversi metodi tra cui: calibrazione, acquisizione grezza e acquisizione rettificata. Il primo metodo è pensato per eseguire la calibrazione del sensore e quindi la stima dei parametri di distorsione cui si è accennato nel capitolo \ref{chap:visione}. Il secondo ed il terzo pensati per richiedere la cattura di un immagine o di una rettificata rispettivamente, utilizzando i parametri ottenuti in fase di calibrazione. Per gestire una camera è necessario quindi implementare tale interfaccia e scrivere il codice necessario solo alla gestione del sensore che si vuole utilizzare. Potrebbe essere quindi necessario, a seconda del sensore, dover scrivere un driver o semplicemente dialogare con uno già esistente. In figura \ref{vis:impl:arch} ad esempio sono stati collegati due sensori USB.
	
	\item \textbf{IEstimator}: è un astrazione dell'algoritmo di stima della matrice di trasformazione che si vuole utilizzare. Espone a monte solo i metodi atti all'introduzione dell'input, richiesta della stima ed estrazione dell'output. Si è previsto quindi di poter cambiare, anche durante l'esecuzione, l'algoritmo usato. Questo meccanismo risulta molto utile soprattutto in fase di test, è possibile sperimentare i risultati ottenuti utilizzando diversi algoritmi. Inoltre, in una futura estensione si potrebbe prescindere dagli algoritmi implementati ed utilizzarne degli altri senza apportare ancora una volta nessuna modifica a monte.
	
	Si sono quindi implementati i diversi algoritmi di stima presentati nel capitolo \ref{chap:stima} ed è possibile scegliere quale utilizzare. Per l'algoritmo di Kabsch ad esempio si è sperimentato con diverse implementazioni, una su CPU e un altra che fa uso del processore grafico.
	
	\item \textbf{ISetsGenerator}: questo blocco funzionale è adibito alla generazione dei set di coordinate dei punti omologhi cui si è discusso nel capitolo \ref{chap:visione}. Accetta in ingresso l'input visivo e restituisce a monte i set suddetti senza esibire nulla riguardo il suo funzionamento interno. In figura \ref{vis:impl:arch} si può evincere che ad esso è collegato un ulteriore blocco funzionale IPairsGenerator. In effetti questo tipo di collegamento non è obbligatorio, diverse implementazioni potrebbero anche farne a meno ma si è preferito segnalarne il possibile collegamento, dato che in effetti nell'implementazione attuale viene utilizzato. Con ordine quindi:
	\begin{itemize}
		\item \textbf{IPairsGenerator}: è un astrazione dell'algoritmo adibito alla ricerca di corrispondenze in due immagini. Concettualmente, accetta due immagini in ingresso e fornisce le corrispondenze tra i punti chiave trovati. Si è implementato quanto detto nel capitolo \ref{chap:visione}: FAST + BRIEF + BF matcher. Si sono previste, scrivendo codice OpenCV, implementazioni che fanno uso sia della vettorizzazione su CPU e/o utilizzano OpenCL. Inoltre per i sistemi dotati di processore grafico Nvidia si è previsto l'utilizzo di un implementazione CUDA.
		\item Inoltre si è implementato un ISetsGenerator che, in concordanza a quanto detto nel paragrafo \ref{sec:vision:solution}, fa uso degli IPairsGenerator forniti al fine di generare i suddetti set.
	\end{itemize}
	Tutto il codice implementato risulta modulare e può essere modificato o sostituito senza necessità di riscrivere il codice di gestione. Visual Observer può essere stanziato specificando quale implementazione utilizzare a seconda delle necessità o delle disponibilità hardware. Per quanto riguarda il funzionamento ad alto livello esso si limita ad utilizzare i livelli sottostanti senza necessità di conoscerne il funzionamento, dimostrando così il potenziale che un architettura a livelli di astrazione può fornire. In particolare, si utilizza il manager per richiedere un nuovo input dall'esterno dai sensori frontali o da una qualsiasi altra coppia sensori. L'input ottenuto viene fornito poi all'ISetsGenerator che restituisce i due set di coordinate. Quest'ultimi vengono infine passati all'IEstimator per ottenere la stima vera e propria che può essere poi fornita a monte ad un qualsivoglia controllore o osservatore.
	 
\end{enumerate}
\section{Validazione e test}
\label{sec:test}

\subsection{Il motore grafico Unity3D}
\label{sec:unity}

\subsection{Vantaggi e svantaggi di una validazione con motore grafico}
\label{sec:valid}

\subsection{Risultati ottenuti}
\label{sec:perf}
%\chapter{Implementazione Framework}
\label{chap:impl}

\begin{minipage}{12cm}\textit{In questo capitolo si illustreranno gli aspetti tecnici del problema in esame: la scelta delle librerie, i problemi riscontrati e le soluzioni intraprese. In particolare si porrà particolare accento sull'efficienza al fine di rendere lo stimatore applicabile in contesti real-time.}
\end{minipage}

\vspace*{1cm}

\section{Strumenti utilizzati}
\label{sec:tools}


\subsection{OpenCV}
\label{sec:tools:opencv}

\subsection{CUDA}
\label{sec:tools:cuda}



\section{Lo sviluppo}
\label{sec:sviluppo}

\subsection{Ambiente di sviluppo e target}
\label{sec:dev:ambiente}

\subsection{Obbiettivi di sviluppo e soluzioni}
\label{sec:dev:obj}



%\chapter{Test}
\label{chap:test}

\begin{minipage}{12cm}\textit{In questo capitolo verranno illustrate le modalità di test e validazione. In particolare verrà usato un motore grafico al fine di generare l'input visuale.}
\end{minipage}

\vspace*{1cm}

\section{Il motore grafico Unity3D}
\label{sec:unity}


\section{Vantaggi e svantaggi di una validazione con motore grafico}
\label{sec:valid}

\section{Risultati ottenuti}
\label{sec:perf}



\chapter{Conclusioni e sviluppi futuri}

In questo lavoro di tesi, si sono toccati diversi ambiti: dalla geometria alla stima, dalla visione stereoscopica fino al riconoscimento delle feature. Si è inoltre effettuato un grosso lavoro di progettazione e implementazione, si sono dovuti prima di tutto individuare i problemi da affrontare per poi, ovviamente, fornirne una soluzione. L'obiettivo iniziale era del tutto differente ed il risultato ottenuto è stato un susseguirsi di idee e tentativi. Talvolta il risultato poteva sembrare troppo lontano, ma procedendo a testa bassa, dividendo il problema in diversi più piccoli, da affrontare uno alla volta, si è riuscito a fornire una soluzione ad ognuno di essi. Si è potuto studiare strumenti innovativi, sperimentarli e talvolta scartarli per arrivare alla fine ad un prodotto finito, funzionante. Sicuramente il lavoro svolto ha ancora grossi margini di miglioramento, si dovrà lavorare sulla stabilità del software, sperimentarlo nel mondo reale e nuove funzionalità o algoritmi potranno essere inseriti. Grazie all'architettura progettata tale aspetto risulterà più semplice, si potranno modificare singole parti del software, fornire implementazioni diverse. Eventualmente, più persone potranno collaborare, migliorare il prodotto insieme e condividere quanto fatto. 

In conclusione, si sono toccate differenti materie al fine di convergere verso un prodotto funzionante, aperto e aggiornabile.

Si è fornito un possibile approccio risolutivo basato sull'utilizzo di strumenti appartenenti a campi differenti: stima, geometria e visione artificiale. Si sono forniti gli strumenti atti a rendere il lavoro svolto comprensibile anche da un lettore non ferrato in questi campi, nella speranza che esso possa essere di ispirazione e possa servire da punto di partenza per sviluppi futuri. 

Il problema è stato diviso in uno di stima e uno di visione. Nel capitolo \ref{chap:stima} si è appunto studiato il primo problema, si sono presentate due differenti soluzioni: la prima basata sull'utilizzo di tecniche algebriche e geometriche. Si è pertanto studiato argomenti non facenti parte del corso di studio, quali le basi di Groebner e la teoria dell'eliminazione. Si è utilizzato il software Macaulay per applicare tali concetti e si è ottenuta una soluzione capace di stimare la matrice di rotazione e il vettore di traslazione nel caso planare. Tale soluzione è stata poi estesa al caso spaziale utilizzando concetti di geometria. In particolare si è dimostrato, in maniera differente e in accordo al risultato di Denavit ed Hartemberg, che un qualsiasi spostamento può essere raggiunto con soli due traslazioni e due rotazioni. Si è poi adattato un algoritmo, detto di Kabsch, utilizzato spesso in ambienti bio-informatici, al fine di ottenere una soluzione, in forma chiusa, più robusta ai disturbi di misura. Se ne è data la prova e nel capitolo \ref{chap:implTest} si mostrato il funzionamento numerico.

Il secondo grande problema, presentato nel capitolo \ref{chap:visione}, ha riguardato la generazione di "riferimenti" da poter utilizzare nella stima. Il problema è stato affrontato utilizzando e studiando tecniche innovative appartenenti al campo della visione artificiale. Infine si è fornita una procedura costruttiva.

Negli ultimi due capitoli si è proceduto a descrivere gli aspetti implementativi e di test delle prestazioni. Si è ottenuto un architettura snella, veloce e modulare. Si sono fornite poi diverse implementazioni, facenti uso di caratteristiche hardware diverse, per gli algoritmi presentati.

In futuro, si prevede di continuare lo sviluppo e di trovare risposte ad ulteriori problemi riscontrati sia in sede teorica che in fase di test. Ad esempio si potrebbe:
\begin{itemize}
	\item Prevedere la possibilità di gestire diversi oggetti dinamici nella scena. Questo potrebbe essere fatto in differenti maniere:
	\begin{itemize}
		\item agendo a livello della stima ed individuando ulteriori algoritmi in grado di risolvere un problema che si potrebbe chiamare multy body Procustes. In cui si deve trovare la disposizione ottima di un gruppo di oggetti posti a confronto.
		\item agendo a livello di identificazione dei punti, escludendo i punti chiave appartenenti ad oggetti ritenuti mobili. Si potrebbe fare tale distinzione utilizzando tecniche di machine learning e riconoscimento degli oggetti.
	\end{itemize}
	\item Costruire un prototipo e sperimentarne il funzionamento in un contesto reale.
	\item Migliorare l'integrazione con Unity che allo stato attuale, per motivi tecnici e prestazionali, è stato utilizzato al solo fine di generare un input visivo offline. Con una migliore integrazione, si potrebbe fare in modo che un eventuale agente possa "navigare" e compiere scelte di controllo all'interno del mondo virtuale.
	\item Migliorare quanto fatto, rendendo il software sempre più veloce e meno affetto da errori.
\end{itemize}








% ELENCO DELLE FIGURE (OPZIONALE)
\addcontentsline{toc}{chapter}{Elenco delle figure}
\listoffigures



% BIBLIOGRAFIA
\addcontentsline{toc}{chapter}{Bibliografia}
\begin{thebibliography}{9}
        \bibitem{bib1}W. Kabsch,
            \emph{``A Solution for the Best Rotation to Relate Two Sets of Vectors, 32, 922-923''},
          Acta Crystallographica, 1976.
        
        \bibitem{bib2}E. Rosten, T. Drummond,
        \emph{``Machine learning for high speed corner detection, vol. 1, 430–443''},
        9th European Conference on Computer Vision, 2006.
       
       	\bibitem{bib3}E. Rosten, R. Porter, T. Drummond,
       	\emph{``Faster and better: a machine learning approach to corner detection, vol. 32, 105-119''},
       	IEEE Trans. Pattern Analysis and Machine Intelligence, 2010.
       	
       	\bibitem{bib4}M. Calonder, V. Lepetit, C. Strecha, P. Fua,
       	\emph{``BRIEF: Binary Robust Independent Elementary Features, vol. 6314, 778-792''},
       	11th European Conference on Computer Vision (ECCV), Heraklion, Crete. LNCS Springer, 2010.
       	
      	\bibitem{bib5} B. Gábor,
      	\emph{``Camera calibration With OpenCV''},
      	http://docs.opencv.org, Calib3d Module, Camera calibration and 3D reconstruction. 
      	
      	\bibitem{bib6} C. Possieri, 
      	\emph{``Algebraic Geometry for Control Problems: from observer design to game theory''},
      	Doctor of philosophy in Computer science, Control and geoinformation, XXIX cycle.
      	
\end{thebibliography}
\end{document}
