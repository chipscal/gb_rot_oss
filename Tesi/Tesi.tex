\documentclass[a4paper, 12pt]{report}
\usepackage[utf8x]{inputenc}
\usepackage[italian]{babel}
%\usepackage[T1]{fontenc}
\usepackage{graphicx}
\usepackage{float}
\usepackage[centertags]{amsmath}
\usepackage{amsfonts}
\usepackage{amssymb}
\usepackage{amsthm}
\usepackage{newlfont}
\usepackage{fancyhdr}
\usepackage{tesisty}

%-------------------------------
% DEFINIZIONE DEGLI ENVIRONMENT
%-------------------------------

\newtheorem{obs}{Osservazione}[section]
\newenvironment{oss}
    {\begin{obs}\begin{normalfont}}
    {\hfill $\square \!\!\!\!\checkmark$ \end{normalfont}\end{obs}}

\newtheorem{pro}{Problema}[chapter]
\newenvironment{prob}
    {\begin{pro}\begin{normalfont}}
    {\hfill $\spadesuit$ \end{normalfont}\end{pro}}

\newtheorem{teor}{Teorema}[section]
\newenvironment{teorema}
    {\begin{teor}\textit }
    {\hfill  \end{teor}}

\newtheorem{defn}{Definizione}[section]
\newenvironment{de}
    {\begin{defn}\begin{normalfont}}
    {\hfill $\clubsuit$ \end{normalfont}\end{defn}}

%-----------------------------
% CONFIGURAZIONE DELLA PAGINA
%-----------------------------

\hfuzz2pt % Don't bother to report over-full boxes if over-edge is < 2pt

\fancypagestyle{plain}{
\fancyhead{}\renewcommand{\headrulewidth}{0pt} } \pagestyle{fancy}
\renewcommand{\chaptermark}[1]{\markboth{\small CAP. \thechapter \textit{ #1}} {} }
\renewcommand{\sectionmark}[1]{\markright{\small  \thesection \textit{ #1}} {} }
\voffset=-20pt    % distanza tra il limite superiore del foglio e l'intestazione
\headsep=40pt     % distanza  l'intestazione ed il testo del corpo
\hoffset=0 pt     % misura equivalente al margine sinistro
\textheight=620pt % altezza del corpo del testo
\textwidth=435pt  % larghezza del corpo del testo
\footskip=40pt    % distanza tra il testo del corpo ed il pie' di pagina
\fancyhead{}      % cancella qualsiasi impostazione per l'intestazione
\fancyfoot{}      % cancella qualsiasi impostazione per il pie' di pagina
\headwidth=435pt  % larghezza del'intestazione e del pie' di pagina
\fancyhead[R]{\rightmark} \fancyfoot[L]{\leftmark}
\fancyfoot[R]{\thepage}
\renewcommand{\headrulewidth}{0.3pt}   % spessore della linea dell'intestazione
\renewcommand{\footrulewidth}{0.3pt}   % spessore della linea del pi�di pagina

\numberwithin{equation}{section}
\renewcommand{\theequation}{\thesection.\arabic{equation}}




%--------------------------
% MODIFICARE DA QUI IN POI
%--------------------------

\begin{document}

\dedicate{Inserire la dedica}

\corso{DELL'AUTOMAZIONE} \titoloTesi{NOME TESI} \anno{xxxx/xxxx}
\relatore{Professore}
 \autore{Candidato}
\correlatore{Correlatore Uno\\ Correlatore Due}

\baselineskip=25pt

\intestazione

%------------------------------------------------
% INTRODUZIONE E RINGRAZIAMENTI (NON MODIFICARE)
%------------------------------------------------

\fancypagestyle{plain}{
\fancyhead{}\renewcommand{\headrulewidth}{0pt} } \pagestyle{fancy}
\renewcommand{\chaptermark}[1]{\markboth{\small Cap. \thechapter \textit{ #1}} {} }
\renewcommand{\sectionmark}[1]{\markright{\small  \S \thesection \textit{ #1}} {} }
\voffset=-20pt                         % distanza tra il limite superiore del foglio e l'intestazione
\headsep=40pt                          % distanza  l'intestazione ed il testo del corpo
\hoffset=0pt                           % misura equivalente al margine sinistro
\textheight=620pt                      % altezza del corpo del testo
\textwidth=435pt                       % larghezza del corpo del testo
\footskip=40pt                         % distanza tra il testo del corpo ed il pie' di pagina
\fancyhead{}                           % cancella qualsiasi impostazione per l'intestazione
\fancyfoot{}                           % cancella qualsiasi impostazione per il pie' di pagina
\headwidth=435pt                       % larghezza del'intestazione e del pie' di pagina
\fancyhead[R]{\rightmark} \fancyfoot[L]{\leftmark}
\fancyfoot[R]{\thepage}
\renewcommand{\headrulewidth}{0.3pt}   % spessore della linea dell'intestazione
\renewcommand{\footrulewidth}{0.3pt}   % spessore della linea del pi�di pagina

\pagenumbering{Roman} \tableofcontents
\newpage

\pagenumbering{arabic}

\fancyhead[R]{Introduzione} \fancyfoot[L]{Introduzione}
\fancyfoot[R]{\thepage}

\chapter*{Ringraziamenti}
\addcontentsline{toc}{chapter}{Ringraziamenti}



\chapter*{Introduzione}
\addcontentsline{toc}{chapter}{Introduzione}

Questo lavoro di tesi nasce, inizialmente, da una richiesta di collaborazione da parte di Thales Alenia. In tale collaborazione si richiedeva lo studio, ed eventualmente la sintesi di una nuova soluzione ad un problema di stima. Quest'ultimo è stato così posto: \textbf{si considerino diverse misure della velocità angolare di un corpo rigido nello spazio, misure effettuate da due sensori differenti posti in posizioni e orientamenti diversi, si richiede, se possibile, di individuare la matrice di rotazione relativa tra il primo ed il secondo. Si deve essere in grado cioè di individuare l'orientamento del secondo sensore rispetto al primo.} Tale problema risulta particolarmente rilevante in ambito satellitare. Spesso infatti, in seguito alla messa in orbita di un satellite, quest'ultimo può subire deformazioni permanenti o transitorie. I sensori a bordo perciò possono risultare ruotati rispetto all'orientamento nominale e fornire una misura di velocità angolare scorretta. Se si individua però la trasformazione avvenuta, si è in grado di correggere la misura.

La prima metà di questo lavoro di tesi si concentrerà sulla soluzione del suddetto problema. Si preannuncia che la domanda posta ha risposta affermativa e si forniranno due soluzioni. In realtà tali soluzioni abbracceranno un problema più ampio, che comprende il primo, nel quale si richiede la stima di una trasformazione completa: matrice di rotazione e vettore di traslazione. Quest'ultimo, nel constesto su detto, potrebbe identificare un bias tra i due sensori. 
La prima soluzione si basa sull'utilizzo di tecniche di \textbf{Geometria algebrica} ed in particolare delle \textbf{basi di Groebner}. La seconda è un adattamento di un algoritmo spesso utilizzato in bio-informatica per il confronto di proteine, detto \textbf{Algoritmo di Kabsch} e garantirà una maggiore robustezza ai rumori in misura.

La seconda metà di questo lavoro invece nasce per curiosità personale ed abbraccia il settore della \textbf{visione artificiale}. Comprese le potenzialità degli strumenti individuati infatti, si è scelto di applicarli per fornire una soluzione ad un problema molto rilevante in contesti di \textbf{robotica mobile}. Si è pertanto progettato, implementato e testato un sistema di stima del posizionamento e orientamento assoluti di un osservatore mobile rispetto all'ambiente circostante, eventualemente ignoto, denominato \textbf{Visual Observer}. Tale strumento fa uso solo di fotocamere (almeno due) e può essere utilizzato in ambienti dove altri tipi di sensori falliscono o in concomitanza a soluzioni più comuni al fine di migliorare le prestazioni di stima. Ad esempio potrebbe essere utilizzato in ambienti al chiuso per calcolare la posizione di un drone.

Nel capitolo \ref{chap:mat} si introdurranno brevemente gli strumenti matematici utilizzati che si suppone possano essere non noti al lettore perché non trattati nel corso di studi.
Nel capitolo \ref{chap:visualObs} si motiverà e si esplicherà il funzionamento del sistema di posizionamento. Si renderà necessario, come vedremo, sia risolvere il problema di stima vero e proprio sia la generazione di un input valido per gli algoritmi individuati. Si preannuncia che sarà necessario: riuscire a misurare le coordinate dei punti del mondo utilizzando tecniche di \textbf{steroscopia}, scegliere un set di punti del mondo e "ricercare" gli stessi in un istante successivo al netto dello spostamento, mediante tecniche di \textbf{Feature Detection}, per procedere alla generazione di un set di coppie di "misure" da utilizzare come input agli algoritmi di stima individuati. Nel capitolo \ref{chap:stima} si risolverà pertanto il primo problema, invece nel \ref{chap:visione} si provvederà a fornire gli strumenti fini alla generazione del suddetto set.

Infine nel capitolo \ref{chap:implTest} si illustreranno le problematiche riscontrate, gli strumenti utilizzati e le soluzioni individuate durante l'implementazione del software. L'intera implementazione sarà improntata nel rispetto di un requisito di esecuzione realtime, si di poter rendere il sistema utilizzabile in contesti di controllo in feedback di robot mobili. Infine si testeranno le prestazioni e si validerà il sistema mediante l'uso di un motore grafico e l'interazione quindi con un ambiente virtuale.  

\fancyhf{} %elimina header/footer vecchi


\fancyhead[R]{\rightmark} \fancyhead[L]{\leftmark}
\fancyfoot[R]{\thepage}





%---------------------
% INCLUSIONE CAPITOLI
%---------------------


\chapter{Capitolo0}
\label{chap:fond}

\begin{minipage}{12cm}\textit{bla bla}
\end{minipage}

\vspace*{1cm}

\section{BlaBla}
\label{sec:BlaBla}





\chapter{Conclusioni e sviluppi futuri}

In questo lavoro di tesi, si sono toccati diversi ambiti: dalla geometria alla stima, dalla visione stereoscopica fino al riconoscimento delle feature. Si è inoltre effettuato un grosso lavoro di progettazione e implementazione, si sono dovuti prima di tutto individuare i problemi da affrontare per poi, ovviamente, fornirne una soluzione. L'obiettivo iniziale era del tutto differente ed il risultato ottenuto è stato un susseguirsi di idee e tentativi. Talvolta il risultato poteva sembrare troppo lontano, ma procedendo a testa bassa, dividendo il problema in diversi più piccoli, da affrontare uno alla volta, si è riuscito a fornire una soluzione ad ognuno di essi. Si è potuto studiare strumenti innovativi, sperimentarli e talvolta scartarli per arrivare alla fine ad un prodotto finito, funzionante. Sicuramente il lavoro svolto ha ancora grossi margini di miglioramento, si dovrà lavorare sulla stabilità del software, sperimentarlo nel mondo reale e nuove funzionalità o algoritmi potranno essere inseriti. Grazie all'architettura progettata tale aspetto risulterà più semplice, si potranno modificare singole parti del software, fornire implementazioni diverse. Eventualmente, più persone potranno collaborare, migliorare il prodotto insieme e condividere quanto fatto. 

In conclusione, si sono toccate differenti materie al fine di convergere verso un prodotto funzionante, aperto e aggiornabile.

Si è fornito un possibile approccio risolutivo basato sull'utilizzo di strumenti appartenenti a campi differenti: stima, geometria e visione artificiale. Si sono forniti gli strumenti atti a rendere il lavoro svolto comprensibile anche da un lettore non ferrato in questi campi, nella speranza che esso possa essere di ispirazione e possa servire da punto di partenza per sviluppi futuri. 

Il problema è stato diviso in uno di stima e uno di visione. Nel capitolo \ref{chap:stima} si è appunto studiato il primo problema, si sono presentate due differenti soluzioni: la prima basata sull'utilizzo di tecniche algebriche e geometriche. Si è pertanto studiato argomenti non facenti parte del corso di studio, quali le basi di Groebner e la teoria dell'eliminazione. Si è utilizzato il software Macaulay per applicare tali concetti e si è ottenuta una soluzione capace di stimare la matrice di rotazione e il vettore di traslazione nel caso planare. Tale soluzione è stata poi estesa al caso spaziale utilizzando concetti di geometria. In particolare si è dimostrato, in maniera differente e in accordo al risultato di Denavit ed Hartemberg, che un qualsiasi spostamento può essere raggiunto con soli due traslazioni e due rotazioni. Si è poi adattato un algoritmo, detto di Kabsch, utilizzato spesso in ambienti bio-informatici, al fine di ottenere una soluzione, in forma chiusa, più robusta ai disturbi di misura. Se ne è data la prova e nel capitolo \ref{chap:implTest} si mostrato il funzionamento numerico.

Il secondo grande problema, presentato nel capitolo \ref{chap:visione}, ha riguardato la generazione di "riferimenti" da poter utilizzare nella stima. Il problema è stato affrontato utilizzando e studiando tecniche innovative appartenenti al campo della visione artificiale. Infine si è fornita una procedura costruttiva.

Negli ultimi due capitoli si è proceduto a descrivere gli aspetti implementativi e di test delle prestazioni. Si è ottenuto un architettura snella, veloce e modulare. Si sono fornite poi diverse implementazioni, facenti uso di caratteristiche hardware diverse, per gli algoritmi presentati.

In futuro, si prevede di continuare lo sviluppo e di trovare risposte ad ulteriori problemi riscontrati sia in sede teorica che in fase di test. Ad esempio si potrebbe:
\begin{itemize}
	\item Prevedere la possibilità di gestire diversi oggetti dinamici nella scena. Questo potrebbe essere fatto in differenti maniere:
	\begin{itemize}
		\item agendo a livello della stima ed individuando ulteriori algoritmi in grado di risolvere un problema che si potrebbe chiamare multy body Procustes. In cui si deve trovare la disposizione ottima di un gruppo di oggetti posti a confronto.
		\item agendo a livello di identificazione dei punti, escludendo i punti chiave appartenenti ad oggetti ritenuti mobili. Si potrebbe fare tale distinzione utilizzando tecniche di machine learning e riconoscimento degli oggetti.
	\end{itemize}
	\item Costruire un prototipo e sperimentarne il funzionamento in un contesto reale.
	\item Migliorare l'integrazione con Unity che allo stato attuale, per motivi tecnici e prestazionali, è stato utilizzato al solo fine di generare un input visivo offline. Con una migliore integrazione, si potrebbe fare in modo che un eventuale agente possa "navigare" e compiere scelte di controllo all'interno del mondo virtuale.
	\item Migliorare quanto fatto, rendendo il software sempre più veloce e meno affetto da errori.
\end{itemize}
\chapter*{Appendice A\\ Questa \`e un'appendice}
\addcontentsline{toc}{chapter}{Appendice A - Questa \`e un'appendice}

%\begin{figure}[POS]
%	\centering
%	\includegraphics[width=Xcm]{imgs/NOME.eps}
%	\caption{Descrizione della figura.}
%	\label{fig-label-figura}
%\end{figure}


\chapter*{Appendice B\\ Seconda appendice}
\addcontentsline{toc}{chapter}{Appendice B - Seconda appendice}





% ELENCO DELLE FIGURE (OPZIONALE)
\addcontentsline{toc}{chapter}{Elenco delle figure}
\listoffigures


% BIBLIOGRAFIA
\addcontentsline{toc}{chapter}{Bibliografia}
\begin{thebibliography}{9}
        \bibitem{bib1}Nome Autore,
            \emph{``Nome del libro''},
          Nome Editore, Anno di Pubblicazione.
        \bibitem{bib2}C. Bonivento - C. Melchiorri - R. Zanasi,
            \emph{``Sistemi di controllo digitale''},
          Progetto Leonardo, 1995.
\end{thebibliography}
\end{document}
