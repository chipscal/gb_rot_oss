\chapter{Visual Observer}
\label{chap:visualObs}

\begin{minipage}{12cm}\textit{In questo capitolo verranno illustrate le motivazioni e le dinamiche che hanno portato alla decisione dell'implementazione di uno stimatore di posizione basato su visione. Inoltre verrà fornita una descrizione generica sulla logica di funzionamento; logica che verrà approfondita nei capitoli successivi.}
\end{minipage}

\vspace*{1cm}



\section{Motivazioni}
\label{sec:motivi}

Una delle problematiche che si deve affrontare se ci si occupa di robotica, in particolare mobile, è sicuramente la stima della posizione del robot stesso. Infatti risulta fondamentale avere un feedback dell'ambiente circostante per poter controllare in modo efficace gli spostamenti di un robot. Si pensi ad esempio ad un drone in volo in ambienti chiusi, o un robot di servizio che deve compiere spostamenti in un abitazione o un ambiente industriale. Talvolta ottenere tale stima risulta più semplice, ad esempio un robot su ruote può misurare il suo spostamento in base al numero di giri delle stesse. Per un robot bipede invece tale approccio risulta più difficile, mentre per un drone aereo sicuramente non praticabile. Ancora una volta, come spesso accade in molti campi ingegneristici, la natura fornisce una soluzione elegante ed efficace. Tale soluzione è basata sulla vista, in particolare la vista stereoscopica. Molti animali utilizzano infatti una coppia di "sensori" capaci di catturare un'immagine stereoscopica del mondo circostante, tale immagine viene poi elaborata a mezzo del sistema centrale per fornire un informazione del mondo circostante e può essere impiegata per prendere delle decisioni. Si può facilmente affermare che tutti gli animali dotati di questa funzionalità hanno un vantaggio competitivo rispetto a quelli che non la possiedono.
Risulta naturale allora cercare di mimare tale funzione. Purtroppo sebbene si conoscano i principi base della tecnologia, l'elaborazione che viene compiuta dal cervello animale è sconosciuta.

Negli ultimi anni si sono compiuti numerosi studi in tale senso, e continuamente vengono presentati nuovi strumenti più efficienti e/o più efficaci. Tipicamente si possono suddividere queste soluzioni in due tipologie: nella prima si fa uso di soli sensori a bordo del robot, invece nella seconda si utilizzano sistemi a costellazioni. Per costellazione si intende una serie di sensori, disposti nel ambiente di lavoro, che forniscono informazioni (anche di diverso tipo) che incrociate forniscono una stima della posizione del robot. Sebbene i secondi tipicamente sono più accurati, risultano limitanti e non applicabili in molte circostanze rispetto ai primi. Ad esempio in ambienti ignoti non è possibile aver predisposto una costellazione.

Questo lavoro di tesi nasce con lo scopo di studiare tale problematica e fornirne una soluzione. L'intero lavoro, dalle scelte dei strumenti teorici da utilizzare all'implementazione, è stato impostato con il duplice obbiettivo, oltre all'ottenimento della stima, di riuscire ad ottenerla in tempo reale. Infatti si è puntato allo sviluppo di un framework che possa essere impiagato in una legge in anello in modo efficace.  

 

\section{Approccio al problema}
\label{sec:approccio}
Nel precedente paragrafo si sono menzionate le motivazioni e i passi logici che hanno portato allo sviluppo di un sistema di stima della posizione attraverso strumenti di visione. Per comprendere il lavoro svolto è bene avere quindi un idea, almeno generale, dei problemi che uno strumento del genere deve affrontare e su come essi siano stati risolti. 

Il primo problema trattato senza dubbio è stata la scelta della struttura e gli strumenti matematici da utilizzare. L'osservatore è strutturato nella seguente maniera: \textit{si supponga l'osservatore capace di compiere movimento (traslazione e rotazione) nello spazio. Istante per istante si procede alla stima della matrice di trasformazione relativa che descrive tale spostamento a partire dalla situazione precedente a quella attuale. Le trasformazioni relative possono essere integrate al fine di ottenerne una assoluta.} 
Il problema allora si riduce alla stima della matrice di trasformazione relativa tra due istanti del tempo. Utilizzando un approccio, divede et impera, il problema si può scindere in due: il primo sotto-problema riguarda la stima della matrice di trasformazione a partire da set di punti omologhi e il secondo la loro generazione. Entrando più nel dettaglio:
\begin{enumerate}
	\item \textbf{Stima della trasformazione}: Si suppongono note le coordinate, nel sistema mobile dell'osservatore, di un certo numero di punti omologhi (cioè le coordinate dello stesso punto prima e dopo la rotazione). Dalla conoscenza di queste, si ottiene una stima della matrice di trasformazione relativa. Questo problema sarà trattato nel capitolo \ref{chap:stima}. Si ritiene opportuno notare che la soluzione a questo problema è di utilità ancora più generale. Infatti, ad esempio, si potrebbe utilizzare lo stesso strumento per stimare la differenza di orientamento tra due sensori atti alla misura della stessa misura, inoltre se ne potrebbe stimare anche il bias relativo.
	
	\item \textbf{Generazione dei set}: Questo problema potrebbe essere affrontato in diversi modi. In questa sede si è scelto di utilizzare un approccio basato su un particolare campo della visione artificiale: il \textbf{Feature Detection}. Come vedremo meglio nel capitolo \ref{chap:visione}, si utilizzano strumenti di individuazione delle feature (cioè punti particolari della scena), in combinazione con ragionamenti di stereoscopia, al fine di costruire i suddetti set. Il vantaggio principale, come sarà più chiaro nel seguito, è che questi punti sono individuabili più facilmente e con meno errore in due scene distinte. Tali scene possono essere sia le due catturare dalla coppia di camere stereo nello stesso istante, sia due scene catturate dalla stessa camera in istanti diversi. Si sarà quindi in grado di calcolare sia le coordinate spaziali di questi punti particolari, sia di "ritrovare" lo stesso punto dopo una rotazione e traslazione al fine di ricavarne una coppia omologa.
\end{enumerate}





